\documentclass[12pt,a4paper,twoside]{article}
\usepackage{bookmark}
\usepackage{amsmath}
\usepackage{parskip}
\usepackage{enumitem}
\usepackage{hyperref}
\urlstyle{same}
\usepackage{xcolor}
\usepackage[multiple]{footmisc}
\usepackage[margin=25mm]{geometry}
% \usepackage[backend=biber]{biblatex}
% \addbibresource{stankeviciute-proposal.bib}

\begin{document}

\begin{center}
\Large
Computer Science Tripos -- Part II -- Project Proposal\\[4mm]
\LARGE
A graph convolutional network for age prediction \\[4mm]

\large
Kamilė Stankevičiūtė (\texttt{ks830}), Gonville \& Caius College

\today % October 2019
\end{center}

\vspace{5mm}
\textbf{Project Originator:} Mr Tiago Azevedo

\textbf{Project Supervisors:} Mr Tiago Azevedo, Prof Pietro Liò

\textbf{Directors of Studies:} Dr Timothy~M.~Jones, Dr Graham~Titmus

\textbf{Project Overseers:} Prof Jon~Crowcroft, Dr Thomas~Sauerwald

% Main document

\section*{Introduction}
% The problem to be addressed.

% [Tiago] Why NNs and not something else? You probably want one sentence of motivation saying they have been very successful in other fields, and then one sentence that as a consequence they might help physicians.
Neural networks provide the opportunity to capture the similarities between patients and trends which might help physicians to understand the mechanisms of the disease and in turn find more effective treatments.

\textit{Give more information on graph neural networks + spectral convolutions}

One way of modelling the population is a graph, with nodes representing individuals (their features and the label), and edges corresponding to associations between individuals according to some heuristic or a formally defined similarity metric. Additionally, the graph structure is helpful when incorporating multiple modalities of data (e.g. imaging and non-imaging).

One reason why such graph representation is considered to be useful is that it makes use of both the individual patient data (node feature vectors) and the trends in the population through pairwise similarities (graph edges), inferring the information about the patient's label from their neighbourhood.

\subsection*{Project description}

This project was inspired by a state-of-the-art Graph Convolutional Network (GCN) as described in papers by Parisot et al. \cite{parisot2017spectral, parisot2018disease} In these papers, the GCN (adapted from Kipf and Welling \cite{kipf2017semi}) was used for two tasks: 1) prediction of Autism Spectrum Disorder from the ABIDE dataset and 2) prediction of a progressive form of Mild Cognitive Impairment that develops into Alzheimer's disease from the ADNI dataset.

\textbf{Data, Models, Evaluation}

\begin{itemize}
  \item Applying GCN to patient populations
  % [Tiago] in order to be easier to read (more fluid) you probably want each word in the beginning of the list to follow the same structure (ie. you are using present continuous in all bullet points except here)
  \item Predicting the subject's age from their brain imaging and other data.
  \item Using PyTorch rather than TensorFlow to make use of the broader range of APIs supporting graph data in PyG—\textit{making the final library more accessible and extensible, contributing to the open-source community} % [Tiago]
  \item \textit{Training and test data}
\end{itemize}

\section*{Starting point}

% Describe existing state of the art, previous work in this area,
%   libraries and databases to be used. Describe the state of any
%   existing codebase that is to be built on.

The source code for the implementation of Kipf and Welling's\cite{kipf2017semi} GCNs and Parisot et al.'s \cite{parisot2018disease} first classification task is publicly available online.\footnote{\url{https://github.com/tkipf/gcn}}\footnote{\url{https://github.com/parisots/population-gcn}}

I will be using PyTorch for this project because of its support for machine learning on structured graph data. In particular, PyTorch Geometric (PyG)\footnote{\url{https://github.com/rusty1s/pytorch_geometric}}—a geometric deep learning extension library—will make the implementation, iteration and extensions to the model more flexible in addition to performance improvements and simplified APIs.

I have experience with the basics of TensorFlow\footnote{Five-course Deep Learning specialisation by deeplearning.ai on Coursera}\footnote{Google's Machine Learning Crash Course and follow-up courses.} and no experience with PyTorch or graph neural networks. I have attended or will study (possibly in advance) the CST courses related to the subject of this project (such as IA Machine Learning and Real-World Data, IB Artificial Intelligence, II Bioinformatics, II Machine Learning and Bayesian Inference).

\section*{Resource declaration}

For the most part of this project I will be using my personal MacBook Pro (2019, with 1.4 GHz Quad-Core Intel Core i5 processor and 8GB of RAM). Training the model will require the use of GPUs provided by the Computational Biology Group (as confirmed by Prof Pietro Liò). To prevent any loss of data, both the source code and \LaTeX\ source will be stored on my machine, private GitHub repositories, and Google Drive, as well as regularly backed up on an external HDD.

\section*{Work to be done}
\label{section:work}

% [Tiago] bullet points should start with the same sentence structure

% Describe the technical work.
List of explicit \textbf{deliverables}.
\begin{itemize}
  \item Preprocess the data (depending on the dataset). For example, ADNI would require fetching the data from the database and preprocessing the brain volumes into $C=138$ longitudinal scans (“large scale segmentation analysis (into 138 anatomical structures using MALP-EM (Ledig et al., 2015))”) as in the Parisot paper. Fetching for the other datasets. \textit{Deliverable}: data processing pipeline. % [Tiago] "data cleaned and is ready for analysis"
  \item Connecting the datapoints into the graph based on similarity criteria as defined in the paper (if the dataset is applicable). \textit{Deliverable}: Datapoints are fetched and connected into the graph based on pairwise similarity. \textit{Deliverable}: definition of the similarity metric/threshold. % [Tiago] deliverable is something specific, not a verb/action
  \item Implement of the GCN that reproduces the results of the paper (depending on the dataset). \textit{Deliverable}: Kipf's GCN (\texttt{GCNConv} in PyG) that can be trained on the graph constructed into the previous point. \textit{Deliverable}: The implementation of the graph neural network with the hyperparameters as listed in the paper (if applicable for the dataset) that reproduces the results.
  % [Tiago] GAT originally doesn't allow for weighted edges. You probably want to say GAT because of interesting results in previous literature. Thus, you can probably divide this point: (1) implementation of another graph NN layer, (2) Include weights (in theory you can even edit the message passing mechanism in GCN to multiply by the weights, just like you are suggesting for GAT) 
  \item An alternative graph neural network for performance comparison. One interesting choice could be a Graph Attention Network (GAT), different from \texttt{GCNConv} due to the use of weighted edges. Note that for this to be used a custom similarity metric should be implemented that gives continuous (rather than binary) edge weights. \textit{Deliverable}: a graph neural network that uses a different approach for convolutions, such as GAT.
  % [Tiago] What exactly would you be testing points? Eg. what a unit test would consist of?
  \item \textit{Testing framework.} This is a good software engineering practice that makes it more likely that the code does not contain significant logic errors or other bugs. An exact testing framework could be developed for the preprocessing part, perhaps some \textit{small} edge cases in the graph neural network could be verified using random seeds. \textit{Deliverable}: Testing framework.
  % [Tiago] I just recalled that one thing we discussed could be how it handles missing data (eg. a certain percentage without some data), which could create an interesting view on robustness and semi-supervised learning. Maybe this could go to extension (or "personal" extension in case you have time and you can say you had one more extension than initially planned)
  \item \textit{Evaluation framework.} One interesting evaluation criterion that is not listed in the Parisot papers which should be implemented in any case (accuracy, AUC), is the measure of \textit{robustness}, which could be defined as the \textit{rate at which the accuracy drops as more information is removed from the nodes}. It is a good way to see how much performance is dependent on the \textit{connections} rather than the \textit{individual patient data}—how much can we infer from the similar neighbours alone without knowing anything about the patient in question. \textit{Deliverable}: evaluation based on the metrics as described in Parisot papers—accuracy, AUC—for direct comparison with the baseline methods. \textit{Deliverable}: an API that takes in the algorithm and the GCN, reports the accuracy measures at each level of removed input data. Should be able to return the actual graph of this accuracy dropping.
  \item (?) Statistical processing (for evaluation).
  \item Writing up the dissertation. \textit{Deliverable}: obvious.
\end{itemize}


% \textit{Rough list of things to do (based on ADNI, change to another dataset if appropriate).}
% \begin{itemize}
%   \item Read Kipf and Welling's GCN paper \cite{kipf2017semi} and blog post. Read about spectral graph theory because that seems to be the basis of graph convolutions [seems difficult to grasp—also revise/study more linear algebra]
%   \item Get familiar with \texttt{torch\_geometric} (PyG), TensorFlow
%   \item Work on fetching and pre-processing the data (e.g. fetch ADNI dataset).
%   \item Get the working implementation for TensorFlow GCN for ADNI dataset, reproduce the results.
%   \item Measure robustness of the original model in TensorFlow GCN
%   \item Write the ADNI implementation using PyG, measure accuracy (aim for at least 80.0\%)
%   \item Find alternative methods in PyG that might work better and improve accuracy
%   \item Measure robustness of the new model—if robustness decreases the new model from previous step might be overfitting
%   \item Incorporate more data from ADNI, if there is too much data consider using autoencoders and other dimensionality reduction techniques.
% \end{itemize}

\section*{Success criteria}

% Describe what you expect to be able to demonstrate at the
% end of the project and how you are going to evaluate your achievement.

% \begin{itemize}
%   \item Reimplement the model in PyTorch (\texttt{pytorch\_geometric}) package with the base accuracy of at least 80.0\% as in \cite{parisot2018disease}
%   \item \textsc{todo}
% \end{itemize}


The above list suggests that implementing the graph convolution network and reproducing the result from Parisot et al. \cite{parisot2018disease} (e.g. 80\% accuracy on ADNI) should be the baseline [\textit{dataset dependent}, might not be applicable]. The extensions to the model should increase those results. [An alternative approach to the baseline GCN should be implemented and their performance should be compared].

Another criterion is that by the end of the project the evaluation framework should be implemented that clearly compares two alternative graph neural network models. An evaluation criterion that could be defined and implemented is \textit{robustness to missing or incorrect data}, revealing how important is the neighbourhood information in accurately predicting the diagnosis compared to the node features only.

\section*{Possible extensions}
% Potential further envisaged evaluation metrics or extensions.
\begin{itemize}
  \item Implement spectral filter computation with \textit{Cayley polynomials} instead of using Chebyshev polynomials. Cayley polynomials have been introduced in a paper by Levie et al. \cite{levie2017cayleynets} and were mentioned in \cite{parisot2018disease} as a possible improvement.
  \item Implement a \textit{custom similarity metric}. Currently the metrics used are defined arbitrarily by the authors based on very few features. Learning a different similarity metric based on more features (such as genetic sequencing—SNPs in APOE gene, TOMM40 gene, whole genome sequencing, more types of MRI or PET images in ADNI) could possibly result in a better performance of the classifier.
\end{itemize}


\section*{Timetable and milestones}
\label{section:timetable}

% A work plan of perhaps ten or so two-week work-packages,
% as well as milestones to be achieved along the way. Provide a
% target date for each milestone.

% [Tiago] you can specify which parts of the work you intend to implement in each 2-week time frame. This will help you having a better idea of how you are keeping up/behind.

%  (01/10/2019 – 16/10/2019)
\textbf{Michaelmas weeks 0-1}

\textit{Preparation.}


\textbf{Milestones.} Submit Phase 1 report by 14/10/2019. Submit draft proposal by 18/10/2019.

% (17/10/2019 – 06/11/2019)
\textbf{Michaelmas weeks 2–4}

\textit{Preparation.}

\textbf{Milestones.} Submit final project proposal by 25/10/2019.

% (07/11/2019 – 20/11/2019)
\textbf{Michaelmas weeks 5–6}

\textit{Core work.}

% (21/11/2019 – 04/12/2019)
\textbf{Michaelmas weeks 7–8}

\textit{Core work.}

\textbf{Michaelmas vacation}

\textit{Core work. Some write-up. Progress report and presentation.}

\textbf{Milestones.} Complete the implementation of the main part of the project.

% (16/01/2020 – 29/01/2020)
\textbf{Lent weeks 0–2}

\textit{Extensions.}
 
\textbf{Milestones.} Submit progress report by 31/01/2020.


% (30/01/2020 – 19/02/2020)
\textbf{Lent weeks 3–5} 

\textit{Extensions, start evaluation.}

% (20/02/2020 – 11/03/2020)
\textbf{Lent weeks 6–8}

\textit{Extensions, evaluation, write-up.}

\textbf{Easter vacation}

\textit{Write-up.}

\textbf{Milestones.} Send out the complete draft for review by 27/03/2020. Submit dissertation early by 20/04/2020.

% (24/04/2020 – 06/05/2020)
\textbf{Easter weeks 0–2}

 Time reserved for any unexpected issues.

% \medskip 
% \printbibliography
\bibliographystyle{unsrt}
\bibliography{stankeviciute_project_proposal}

\end{document}
