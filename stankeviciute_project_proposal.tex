\documentclass[12pt,a4paper,twoside]{article}
\usepackage{bookmark}
\usepackage{amsmath}
\usepackage{parskip}
\usepackage{enumitem}
\usepackage{hyperref}
\usepackage[margin=25mm]{geometry}
% \usepackage[backend=biber]{biblatex}
% \addbibresource{stankeviciute-proposal.bib}

\begin{document}

\begin{center}
\Large
Computer Science Tripos -- Part II -- Project Proposal\\[4mm]
\LARGE
A graph convolutional network for Alzheimer's disease classification\\[4mm]

\large
Kamilė Stankevičiūtė (\texttt{ks830}), Gonville \& Caius College

\today % October 2019
\end{center}

\vspace{5mm}
\textbf{Project Originator:} Tiago Azevedo

\textbf{Project Supervisor:} Tiago Azevedo, Prof Pietro Liò

\textbf{Directors of Studies:} Dr T.~M.~Jones, Prof P.~Robinson, Dr G.~Titmus

\textbf{Project Overseers:} 

% Main document

\section*{Introduction}
% The problem to be addressed.

\textsc{todo:} \textit{introduction to Alzheimer's disease}

The use of neural networks provides the opportunity to capture the similarities between patients and trends which might help physicians to understand the mechanisms of the disease and in turn find more effective treatments. 

One way of modelling the population with some patients having Alzheimer's disease (AD) is a graph, with vertices representing individuals and their features (including the diagnosis), and nodes corresponding to associations between individuals according to some heuristic or a formally defined similarity metric. Additionally, the graph structure is helpful when incorporating multiple modalities of data (e.g. imaging and non-imaging).

One reason why such graph representation is considered to be useful is that it makes use of both the individual patient data (vertex feature vectors) and the trends in the population or pairwise similarities between different patients (graph edges), inferring the patient's diagnosis from their neighbourhood. 

This project will explore such graph convolutional networks (GCNs) as well as alternative geometric deep learning approaches to classify the two forms of Mild Cognitive Impairment (MCI): \textit{stable} and \textit{progressive} (progressing to AD).

\section*{Starting point}

% Describe existing state of the art, previous work in this area,
%   libraries and databases to be used. Describe the state of any
%   existing codebase that is to be built on.

This project will be based on a state-of-the-art graph convolutional network (GCN) as described in papers by Parisot et al. \cite{parisot2017spectral, parisot2018disease} In these papers, the GCN (adapted from Kipf \& Welling \cite{kipf2017semi}) was used for two classification tasks: 1) whether a patient is  healthy or suffering from Autism Spectrum Disorder) and 2) whether a patient having Mild Cognitive Impairment has a stable form or the one progressing to Alzheimer's disease. The GCN has achieved the accuracies of 70.4\% and 80.0\% for the two tasks respectively. 

The source code of this paper (written in TensorFlow) is publicly available at \url{github.com/parisots/population-gcn}. It is fully implemented for the first classification task and not the second. This will be used as a basis for replication of the results on PyTorch (or DGL) and extending the model additional improvements. PyTorch has been chosen for its libraries specialised for machine learning on structured graph data—particularly the \texttt{torch\_geometric} (PyG) package—which will make iteration and extensions to the model more flexible as well as improve its performance and simplify the APIs.

This project will use the ADNI dataset (same as in \cite{parisot2017spectral,parisot2018disease}). The database is available at \url{http://adni.loni.usc.edu}.

I am not particularly familiar with any of the frameworks above (no experience with PyTorch, PyG or DGL but basic TensorFlow experience doing Andrew Ng's deep learning courses) or graph neural networks—a lot of learning to do! :)

\section*{Resources required}

For the most part of this project I will be using my personal MacBook Pro (2019, quad-core 1.4 GHz Intel Core i5 processor, 8 GB LPDDR3 RAM) running macOS Catalina. Training the model will most likely require the use of GPUs provided by the Computational Biology Group (as confirmed by Prof Pietro Liò).

The following measures will be taken to store the work and reduce the likelihood of any loss of data: 
\begin{itemize}
  \item Saving the source code and \LaTeX\ source of the project on my machine, GitHub repositories, Google Drive (and possibly MCS).
  \item Regularly backing up the contents of my laptop on an external HDD.
\end{itemize}

The data required for training (ADNI database) is available on \url{http://adni.loni.usc.edu}, for which I requested and was granted access.

\section*{Work to be done}
\label{section:work}

\textsc{todo}: decide on the main work/extensions split of the tasks below, write a coherent section.

% Describe the technical work.
The tasks shall start from this (semi-ordered) list which can then be allocated to the slots in the \hyperref[section:timetable]{Timetable}. 

This list includes both the core work and the possible extensions (not sure about the split, but this hopefully is enough work). If this is not enough work some further ideas are in the Markdown document.

\begin{itemize}
  \item (Project proposal)
  \item Read Kipf and Welling's GCN paper \cite{kipf2017semi} and blog post. Read about spectral graph theory because that seems to be the basis of graph convolutions [seems difficult to grasp—also revise/study more linear algebra]
  \item Get familiar with \texttt{torch\_geometric} (PyG), DGL, TensorFlow
  \item Write the ADNI data fetcher
  \item Get the working implementation for TensorFlow GCN for ADNI dataset, reproduce results (80.0\% accuracy)
  \item Measure robustness of the original model in TensorFlow GCN
  \item Write the ADNI implementation using PyG, measure accuracy (aim for at least 80.0\%)
  \item Find alternative methods in PyG that might work better and improve accuracy
  \item Measure robustness of the new model—if robustness decreases the new model from previous step might be overfitting
  \item Incorporate more data from ADNI, if there is too much data consider using autoencoders and other dimensionality reduction techniques.
  \item Learn the custom similarity metric instead of using the one arbitrarily defined in \cite{parisot2017spectral, parisot2018disease}
  \item Improve[?] the convolution filters: use Cayley polynomials instead of Chebyshev polynomials [easy to replace, difficult to understand?]
\end{itemize}

\section*{Success criteria}

% Describe what you expect to be able to demonstrate at the
% end of the project and how you are going to evaluate your achievement.

% \begin{itemize}
%   \item Reimplement the model in PyTorch (\texttt{pytorch\_geometric}) package with the base accuracy of at least 80.0\% as in \cite{parisot2018disease}
%   \item \textsc{todo}
% \end{itemize}

\textsc{todo}: define more/more specific criteria

The above list suggests that reproducing the result from \cite{parisot2018disease} (80\% accuracy and probably similar AUC) should be the baseline. Then extensions to the model should increase those results.

Another metric that could be defined is \textit{robustness to missing or incorrect data}, revealing how important is the neighbourhood information in accurately predicting the diagnosis compared to the vertex features only.

\section*{Possible extensions}

% Potential further envisaged evaluation metrics or extensions.

% Given the main success criteria have been achieved early and there is time left, some possible extensions include: 
% \begin{itemize}
%   \item \textsc{todo}
% \end{itemize}

\textsc{todo}: see \hyperref[section:work]{work to be done}.

\section*{Timetable}
\label{section:timetable}

% A work plan of perhaps ten or so two-week work-packages,
% as well as milestones to be achieved along the way. Provide a
% target date for each milestone.

Planned starting date is 10/10/2019.

\textsc{todo} once I finalise \textit{what exactly} do I want to do! The slots are as suggested in the proposal template. I have added the deadlines as \textsc{milestones}.

\begin{enumerate}[leftmargin=*, align=left, font=\bfseries]
\item[10/10/2019—16/10/2019] \hfill \smallskip

\textsc{milestone}: submit Phase 1 report by 14/10/2019

\textsc{milestone}: submit draft proposal by 18/10/2019

\item[17/10/2019—06/11/2019] (Michaelmas weeks 2--4) \hfill \smallskip 

\textsc{milestone}: submit final project proposal by 25/10/2019

\item[07/11/2019—20/11/2019] (Michaelmas weeks 5--6) \hfill \smallskip
 


\item[21/11/2019—04/12/2019] (Michaelmas weeks 7--8) \hfill \smallskip
 


\item[Michaelmas vacation] \hfill \smallskip



\item[16/01/2020—29/01/2020] (Lent weeks 0--2) \hfill \smallskip
 
\textsc{milestone}: submit progress report by 31/01/2020

\item[30/01/2020—19/02/2020] (Lent weeks 3--5) \hfill \smallskip



\item[20/02/2020—11/03/2020] (Lent weeks 6--8) \hfill \smallskip


\item[Easter vacation] \hfill \smallskip


\item[24/04/2020—06/05/2020] (Easter term 0--2)  \hfill \smallskip

\textsc{milestone}: submit the dissertation by 08/05/2020


\end{enumerate}

% \medskip 
% \printbibliography
\bibliographystyle{unsrt}
\bibliography{stankeviciute_project_proposal}

\end{document}
