% Template for a Computer Science Tripos Part II project dissertation
\documentclass[12pt,a4paper,twoside, hidelinks]{report}
\usepackage[export]{adjustbox}
\usepackage[hyphens]{url}
\usepackage[pdfborder={0 0 0}]{hyperref}
\urlstyle{same}
\usepackage[margin=25mm]{geometry}

\usepackage[UKenglish]{babel}
% \usepackage{bibunits}


\usepackage[skip=2pt]{caption}
\usepackage{verbatim}
\usepackage{docmute} % only needed to allow inclusion of proposal.tex

\usepackage{titlesec}
\titleformat{\chapter}[display]   
{\normalfont\huge\bfseries}{\chaptertitlename\ \thechapter}{20pt}{\Huge}   
\titlespacing*{\chapter}{0pt}{10pt}{30pt}
\titleformat{\chapter}{\normalfont\huge\bfseries}{\thechapter}{20pt}{\Huge}

\usepackage{bookmark}
\usepackage{parskip}
\usepackage{enumitem}
\usepackage{tabularx}
\usepackage{longtable}
\usepackage{multirow}

\usepackage{pdfsync}

\usepackage{xcolor}
\usepackage[multiple]{footmisc}

\usepackage{minted}
\usepackage[]{caption}
\newenvironment{code}{\captionsetup{type=listing}}{}

\usepackage{amsmath,amsthm, bm}
\usepackage{mathtools}
\usepackage{blkarray}
\usepackage{amsfonts} 
\usepackage{graphicx} % allows inclusion of PDF, PNG and JPG images
\usepackage{epstopdf}
% \usepackage{auto-pst-pdf} 

\usepackage{rotating}

\graphicspath{{./figs/}}

\raggedbottom                           % try to avoid widows and orphans
\sloppy
\clubpenalty1000%
\widowpenalty1000%

\renewcommand{\baselinestretch}{1.1}    % adjust line spacing to make
                                        % more readable
% \renewcommand{\labelenumii}{\theenumii}
% \renewcommand{\theenumii}{\theenumi.\arabic{enumii}.}
\usepackage{chngcntr}

\renewcommand{\eqref}[1]{\ref{#1}}

\usepackage[backend=bibtex, style=numeric,sorting=none]{biblatex}
\addbibresource{references.bib}

\begin{document}
\counterwithin{listing}{chapter}


%TC:group table 0 1
%TC:group tabular 1 1
%TC:group longtable 1 1


%TC:ignore
%%%%%%%%%%%%%%%%%%%%%%%%%%%%%%%%%%%%%%%%%%%%%%%%%%%%%%%%%%%%%%%%%%%%%%%%
\pagestyle{empty}
\thispagestyle{empty}
\pagenumbering{roman}

\rightline{\large{Kamilė Stankevičiūtė}}

\vspace*{60mm}
\begin{center}
\LARGE
\textbf{Graph neural networks for age prediction from neuroimaging data} \\[5mm]
\large
Computer Science Tripos -- Part II \\[5mm]
Gonville \& Caius College \\[5mm]
2020
\end{center}

%%%%%%%%%%%%%%%%%%%%%%%%%%%%%%%%%%%%%%%%%%%%%%%%%%%%%%%%%%%%%%%%%%%%%%%%%%%%%%
% Proforma, table of contents and list of figures

\pagestyle{plain}
\newpage
\chapter*{Declaration of originality}

I, Kamilė Stankevičiūtė of Gonville \& Caius College, being a candidate for Part II of the Computer Science Tripos, hereby declare that this dissertation and the work described in it are my own work, unaided except as may be specified below, and that the dissertation does not contain material that has already been used to any substantial extent for a comparable purpose.

\bigskip
\leftline{Signed \textit{Kamilė Stankevičiūtė}}

\medskip
\leftline{Date \today}

\chapter*{Proforma}

\begin{tabular}{ll}
Candidate number:   & ?                  \\
Project title:      & Graph neural networks for age prediction from neuroimaging data \\
Examination:        & Computer Science Tripos -- Part II 2020 \\
Word count:         & ? \footnotemark[1] \\
Lines of code:      & ?  \\
Project originator: & Mr Tiago Azevedo                        \\
Supervisors:        & Mr Tiago Azevedo, Mr Alexander Campbell \\ 
\end{tabular}
\footnotetext[1]{Counted using the \LaTeX\ Utilities extension in Visual Studio Code.}
\stepcounter{footnote}

% It is quite in order for the Proforma to point out how ambitious the original aims were and how the work completed represents the triumphant consequence of considerable effort against a background of unpredictable disasters. The substantiation of these claims will follow in the rest of the dissertation.

\section*{Original aims of the project}
% At most 100 words describing the original aims of the project.

To implement the population graph paradigm in modelling multi-modal structural  and functional brain imaging data along with phenotype information of individual patients. Implement and apply different graph neural network architectures – Graph Convolutional Network (GCN) and Graph Attention Network (GAT) – for a patient age prediction task. Evaluate the robustness of graph neural networks to noisy and missing data, comparing their performance against each other and with other known benchmarks.

\section*{Work completed}
% At most 100 words summarising the work completed.

\section*{Special difficulties}

None.

\tableofcontents

\newpage
\section*{Acknowledgements}

%%%%%%%%%%%%%%%%%%%%%%%%%%%%%%%%%%%%%%%%%%%%%%%%%%%%%%%%%%%%%%%%%%%%%%%
%TC:endignore

\pagestyle{headings}
\chapter{Introduction}
\pagenumbering{arabic}

% The Introduction should explain the principal motivation for the project. Show how the work fits into the broad area of surrounding Computer Science and give a brief survey of previous related work. It should generally be unnecessary to quote at length from technical papers or textbooks. If a simple bibliographic reference is insufficient, consign any lengthy quotation to an appendix.

% ~500 words

Many common neurological and neurodegenerative disorders, such as Alzheimer’s disease, schizophrenia and multiple sclerosis, have been associated with abnormal patterns of apparent ageing of the brain~\cite{kaufmann2019}. This link has inspired numerous studies in \textit{brain age estimation} using neuroimaging data~\cite{franke2019ten}. The \textit{brain age gap}, defined as the discrepancy between the estimated brain age and the true chronological age, is a powerful biomarker not only for understanding the biological pathways behind the ageing process, but also for assessing an individual’s risk to various brain disorders and identifying new personalised treatment strategies.


Numerous studies exist applying machine learning algorithms to the problem of brain age estimation, typically using structural magnetic resonance imaging (MRI) and genetic data~\cite{franke2019ten}. They tend to predict the brain age for each sample independently from the others, modelling healthy controls separately from individuals with brain disorders, and often developing separate models for each sex~\cite{kaufmann2019,niu2019improved}. They tend not to explicitly consider patterns that are shared across the subgroups of subjects, training on the full dataset but controlling for confounding effects. Moreover, they rarely include other important brain imaging modalities such as functional MRI (fMRI) time-series data, or clinical expertise of neurologists and psychiatrists, even though a combination of different modalities has been shown to improve the results~\cite{niu2019improved}. 


This dissertation proposes a pipeline that can flexibly combine the richness of minimally preprocessed neuroimaging as well as non-imaging modalities to predict the brain age. The data is combined using the \textit{population graph} representation of patients, where the nodes of the population graph contain subject-specific neuroimaging data, and edges represent pairwise subject similarities using non-imaging data. In addition to controlling for confounding effects, these similarities exploit the neighbourhood information when predicting node labels – an approach which has successfully been applied to a variety of problems in both medical and non-medical domains~\cite{parisot2018disease}. 


The population graph is used as input to two types of neural networks that can operate on graph-structured data – the \textit{graph convolutional}~\cite{kipf2017semi} and the \textit{graph attention}~\cite{velickovic2018graph} networks – to predict the brain age. This follows closely the approach presented in the work of Parisot et al.~\cite{parisot2017spectral,parisot2018disease}, where population graphs are used to classify patients as healthy or suffering from autism spectrum disorder and Alzheimer's disease.


Unlike in other state-of-the-art models, graph neural networks trained on population graphs have the potential to learn from the entire cohort of healthy and affected subjects of both sexes at once, capturing a wide range of confounding effects~\cite{ruigrok2014meta} and detecting the variation in brain age trends of different sub-populations of subjects. 

% \section{Related work}
% \subsection{Population graphs}
% Semisupervised population graphs in Parisot et al. for classifying healthy patients and patients having Mild Cognitive Impairment/Alzheimer's or autism spectrum disorder. These achieve state-of-the-art accuracies around 70\% even for a binary classification task which indicates that brain conditions are generally a complex task. Due to the different nature of the task and different evaluation metrics, this performance cannot be directly compared to the predictive power of the brain age regression task.

% \subsection{Graph neural networks}

% \subsection{Brain age prediction from neuroimaging data}
% Most of current machine learning-based methods for brain age gap prediction work on per-brain basis and do not consider pairwise similarities between patients, not taking into account the population as a whole.

% Graphs are static so it makes sense that most practical approaches would not use this as their model, because it prevents using applying it for new patients

% Kaufmann et al.~\cite{kaufmann2019} uses gradient boosting based techniques (XGBoost)~\cite{chen2016xgboost} for brain age gap prediction from structural magnetic resonance imaging (MRI) data. The study uses a significantly larger dataset of 45,000 people and presents separate models for female and male brain age gap prediction, without considering any pairwise similarities between individuals.

% Another framework for brain age gap prediction is based on Gaussian Processes regression, using raw T1-weighted MRI scans, segmenting them and using principal components analysis (PCA) for dimensionality reduction~\cite{cole2018brain}.


\chapter{Preparation}
% Principally, this chapter should describe the work which was undertaken before code was written, hardware built or theories worked on. It should show how the project proposal was further refined and clarified, so that the Implementation stage could go smoothly rather than by trial and error.
% Throughout this chapter and indeed the whole dissertation, it is essential to demonstrate that a proper professional approach was employed.
% The nature of this chapter will vary greatly from one dissertation to another but, underlining the professional approach, this chapter will very likely include a section headed “Requirements Analysis” and incorporate other references to software engineering techniques.
% The chapter will cite any new programming languages and systems which had to be learnt and will mention complicated theories or algorithms which required understanding.
% It is essential to declare the Starting Point (see Section 7). This states any existing codebase or materials that your project builds on. The text here can commonly be identical to the text in your proposal, but it may enlarge on it or report variations. For instance, the true starting point may have turned out to be different from that declared in the proposal and such discrepancies must be explained.

\section{Data preprocessing}
UK Biobank data preprocessed by Dr Richard Bethlehem.

Describe functional, structural and phenotype features. Brain parcellation atlases. Correlation matrices. Patient exclusion and preprocessing.

Garbage in, garbage out. Data preprocessing unsurprisingly turned out to be the most difficult and the most important data processing task.

\subsection{Structural data}
Describe various techniques, e.g. normalisation,...

\subsection{Functional data}
\textit{Turns out functional data is not as effective as structural brain imaging data.}

Description of correlation matrix computation.

\subsection{Phenotype data}

\section{Population graphs}
Mathematical formulation of the graphs.

\subsection{Similarity metrics}

\subsection{Computational model}
Train/validation/test split, cross-validation, patient selection and exclusion from results, stratification, graph representation (edge lists, node features, edge features,...)

\section{Multilayer perceptrons}
The multilayer perceptron maths, if appropriate.

\section{Graph convolutional networks}

\section{Graph attention networks}

\section{Requirements analysis}

Tasks to be implemented (according to proposal: work to be done, success criteria, possible extensions), their relative importance (priority) and difficulty. Provide the order in which the tasks should be carried out to show good planning skills and account for the changes in proposal where the preprocessing pipeline turned out to be more important than the neural network implementation.

\section{Software engineering practice}
Implementing a flexible preprocessing pipeline which could be customised in the future for a variety of machine learning tasks even outside graph neural networks (a package).

Modular structure encapsulating specific task and having well defined documentations of the others.

Description of software engineering techniques: planning out and executing the project based on requirements analysis, setting tasks, and smoothly meeting the success criteria.

Code reuse (of open source well tested libraries), follow documentation and follow the PEP-8 style guide (or whatever PyCharm encourages).

Incremental development.

Modular structure: e.g. data processing, graph construction, graph neural network modules, robustness evaluation framework. Figure out where validation and cross validation sections should be (while training, separately etc.)

Diagram of the pipelines and module interaction (like in google design docs)


\section{Choice of tools}
PyTorch, PyTorch geometric extension, graph spectral filters/convolutions, message passing, timeseries preprocessing into correlation matrices, IDEs, backup strategies

\section{Starting point}


\chapter{Implementation}
\label{chapter:implementation}
% This chapter should describe what was actually produced: the programs which were written, the hardware which was built or the theory which was developed. Any design strategies that looked ahead to the testing stage might profitably be referred to (the professional approach again).
% Descriptions of programs may include fragments of high-level code but large chunks of code are usually best left to appendices or omitted altogether. Analogous advice applies to circuit diagrams.
% Draw attention to the parts of the work which are not your own. The Implementation Chapter should include a section labelled ”Repository Overview”. The repository overview should be around one page in length and should describe the high-level structure of the source code found in your source code Repository. It should describe whether the code was written from scratch or if it built on an existing project or tutorial. Making effective use of powerful tools and pre-existing code is often laudable, and will count to your credit if properly reported.
% It should not be necessary to give a day-by-day account of the progress of the work but major milestones may sometimes be highlighted with advantage.

%  ~4,500 words

% Tangent works better than correlation or partial correlation.
\section{Overview}

This project can be divided into five key components, as illustrated in Figure~\ref{pipeline-overview}:
\begin{enumerate}
    \item Preparation of the United Kingdom Biobank (UKB) dataset;
    \item Intermediate population graph construction;
    \item Population graph transformation for training;
    \item Training on graph neural network architectures;
    \item Evaluation of the graph neural network performance.
\end{enumerate}

\begin{figure}[h]
    \centering
    \includegraphics[width=\textwidth]{pipeline_overview.pdf}
    \caption{Overview of the key project components.}\label{pipeline-overview}
\end{figure}

The work was split into these particular components so that each of them can carry out a single task independently of the other parts of the program (other than the clearly defined input/output communication), following requirement R3 (see Section~\ref{section:requirements-analysis}). This makes the overall project easier to implement, understand and extend in the future, generalising it to other datasets and preprocessing methods. 

This chapter will explain in detail the implementation behind each of the components.

\section{UKB preprocessing component}
\label{section:ukb-preprocessing}

The main function of the UKB preprocessing component is to prepare the raw or partially preprocessed UKB data for population graph construction. In particular, to improve the computational efficiency of operations later in the pipeline (potentially saving hours or days of computation time), this component preprocesses the data in a convenient form. This involves filtering the dataset, precomputing similarity matrices and functional connectivity matrices. The schematic diagram of these steps is shown in Figure~\ref{preprocessing-component}, which will be referred to throughout this section.

\begin{figure}[h]
    \centering
    \includegraphics[width=\textwidth]{preprocessing_component.pdf}
    \caption{\textit{Overview of the UKB preprocessing component.} 
    This component precomputes the similarity matrices from non-imaging data, and functional connectivity matrices from fMRI data. A unifying subject index is collected to keep track of the subjects across modalities and filter the data for the graph construction component.
    }\label{preprocessing-component}
\end{figure}

\subsection{Cleaning the dataset}
The different modalities of the UKB data, shown in the white box in Figure~\ref{preprocessing-component}, have been provided separately from each other, with a small minority of subjects having the data available for some of the modalities but not others (for example, due to data corruption or its retraction by the participant). To ensure consistent and smooth processing, the subjects with partially missing data (236 in total) have been excluded from further processing. The remaining 17,314 subjects with well-defined modalities have been collected into a \textit{subject index} (bottom left of Figure~\ref{preprocessing-component}), which was used to select, filter and index the data in the subsequent components of the pipeline.

% TODO rb643: as mentioned in another comment I think in future we can do a bit more data cleaning and include additional features in your similarity computation
% TODO rb643: In future implementations it may be worth considering either cortex only or split matrices for cortical and subcortical

\subsection{Precomputing connectivity matrices}
The connectivity matrices involve computing pairwise correlations of 376 time-series (for 360 cortical and 16 subcortical regions) for every subject. With 20 gigabytes of raw time-series data across over 17,000 subjects, this introduces a high computational overhead (a few hours on a CPU\footnote{Measured on a Computational Biology lab machine.} per population graph) if the matrices have to be recomputed every time the graph is constructed. To avoid this, the matrices are computed once for every subject, flattened and their lower triangles stored as \texttt{numpy} arrays as part of the preprocessing component.

% could include the maths here?

\subsection{Precomputing similarity matrices}
The computation of pairwise similarity scores is quadratic in both time and space with respect to the number of subjects. Depending on the exact method of how the similarity function is computed, a general-purpose (non-accelerated) processor might take hours or even days to process the entire dataset. The repeated computation would also be wasteful, since the similarities for a single non-imaging feature do not change over different population graphs: the variation comes from different \textit{selections} of subjects and non-imaging features, their relative weighting, and similarity thresholds.

The similarity matrices – one for each non-imaging feature in Table~\ref{table:phenotype-features} – have therefore been computed in advance. Unlike the functional connectivity data that is used directly for population graph node features (and is flattened and reduced to avoid redundancy), the feature-wise similarity matrices are sliced and filtered depending on the selection of subjects. In this case, it is more practical to store the full matrix, so that the integer indices into the similarity matrix directly correspond to the subject index in other components of the population graph data structure (see Sections~\ref{section:ukb-preprocessing} and~\ref{section:population-graph-representation}).

Having computed the feature-wise similarity matrices, their linear combination for a full similarity score (by default adding matrices together and dividing the result by a constant) can efficiently make use of vectorised matrix operations.

For the \texttt{ICD10} metric, the subjects were considered to be \texttt{ICD10}-similar whenever they had at least one shared mental health or nervous system diagnosis. Two patients without any mental health or nervous system diagnoses were \textit{not} considered to be similar, however, since similarity in having a healthy brain is extremely common (thus not in itself informative) while causing memory issues (see Section~\ref{section:memory}). 

The similarity computation was vectorised in order to make use of hardware acceleration and reduce the compute time: for the boolean \texttt{ICD10}-lookup matrix $\mathbf{F}_{\text{ICD10}}$ with rows indexed by subjects and columns by relevant \texttt{ICD10} diagnoses, the pairwise similarity matrix $\mathbf{M}_{\text{ICD10}}$ computation corresponds to 

\begin{equation}
    \mathbf{M}_{\text{ICD10}} = \mathbf{1}\left[\mathbf{F}_{\text{ICD10}}^{\ }\mathbf{F}_{\text{ICD10}}^{\mathrm{T}} \geq 1\right]
\end{equation}

with the indicator function $\mathbf{1}[\cdot]$ applied element-wise.

For the remaining metrics (e.g.\ years of full-time education, \texttt{FTE}) there is only one integer or floating-point value per subject, with values  compared for equality. The computation is vectorised by exploiting tensor broadcasting semantics\footnote{\url{https://pytorch.org/docs/stable/notes/broadcasting.html}} that copy rows and columns as necessary for the matrix dimensions to match. For the vector of subject \texttt{FTE}s, $\mathbf{f}_{\text{FTE}}^{\mathrm{T}} \in \mathbb{R}^{N \times 1}$ and $\mathbf{F}_{\text{FTE}} = [\mathbf{f}_{\text{FTE}}^{\mathrm{T}} \cdots \mathbf{f}_{\text{FTE}}^{\mathrm{T}}] \in \mathbb{R}^{N \times N}$, \texttt{FTE}-similarity matrix is defined as

\begin{equation}
    \mathbf{M}_{\text{FTE}} = \mathbf{1}\left[\mathbf{F}_{\text{FTE}}^{\ } = \mathbf{F}_{\text{FTE}}^{\mathrm{T}} \right].
\end{equation}

\section{Graph construction component}
\label{section:graph-construction}
The next stage of the pipeline involves constructing the ``intermediate representation'' of the population graph. The intermediate representation contains the graph topology and node features, but is not prepared for training – it is not split into training, validation and test sets, and its features are not normalised. The two stages are separate because this allows for the reuse of the same intermediate representation for different dataset splits and other training parameters without having to reconstruct the edges in $O(N^2)$ time for $N$ subjects. The steps for the data processing in this component are schematically visualised in Figure~\ref{graph-construction-component}.

\begin{figure}[h]
    \includegraphics[width=\textwidth]{graph_construction_component.pdf}
    \caption{\textit{Overview of the graph construction component.}
    The raw and preprocessed UKB data is collected according to the graph specification, and combined into the intermediate population graph representation containing the different modalities of brain imaging features, the node labels and the edges.
    }\label{graph-construction-component}
\end{figure}

\subsection{Inputs}
The inputs to the graph construction component can be categorised into four types as shown in the box at the bottom left of Figure~\ref{graph-construction-component}:
\begin{enumerate}
    \item \textit{Modality specification} describes which of the neuroimaging modalities should be used as node features. This could be any combination of functional, structural and quality control data, following the pipeline flexibility requirement R2 (Section~\ref{section:requirements-analysis}).
    \item \textit{Subject specification} is an optional argument overriding the default option to use the entire dataset (filtered by the subject index described in the previous component) when constructing the population graph. In this case, a number of subjects or a list of particular UKB identifiers could be provided.
    \item \textit{Similarity specification}. 
    In its default implementation, the similarity score is computed as the average over a set of similarity features $\{M_1, \dots, M_n\}$ (see Equation~\eqref{eq:similarity}), in which case it is sufficient to specify the similarity feature set to be used. An extension to this is to accept an arbitrary linear combination of various similarity features, allowing for much richer similarity metrics, again increasing the flexibility of the pipeline (see Requirement R2 in Section~\ref{section:requirements-analysis}). 
    \item \textit{Similarity threshold}. A number $\mu \in [0,1]$ defining the threshold for the similarity metric above which an edge will be added to the graph (see Equation~\eqref{eq:similarity-threshold}).
\end{enumerate}


\subsection{Imaging data collection}

Based on the subject and modality specification, the relevant imaging data is collected from the raw UKB files (first filtered by subject index) and stored in the intermediate representation as a \textit{dataframe} (\texttt{pandas.DataFrame} object) indexed by UKB subject identifier. This is represented by the connections between UKB preprocessing component, graph specification and the blue imaging data collection box in Figure~\ref{graph-construction-component}. 

If a particular modality is unused, the dataframe stored is empty.

\subsection{Edge construction}

The edge construction component uses the similarity specification and the filtered similarity matrix data to find the similarity scores for each pair of subjects determined by the subject specification. Whenever the similarity score between two subjects exceeds the threshold, an undirected edge is added to the population graph. The edges are stored in a \textit{tensor} (\texttt{torch.tensor}), a PyTorch datatype for multi-dimensional matrices\footnote{\url{https://pytorch.org/docs/stable/tensors.html}}.

\subsection{Brain health mask computation}

Following the brain age estimation method discussed in Section~\ref{brain-age-estimation}, the machine learning model can only be trained on subjects with healthy brains, although the population graph may contain both healthy and non-healthy subjects. The \textit{brain health mask} is therefore computed to determine which subjects can be used to train the model and which cannot. In particular, non-healthy subjects are not included in loss function computation, so that the direction of parameter update depends on healthy subjects only (though the parameters are updated for the entire graph). As a result, the age predictions are available for both healthy and non-healthy subjects, but the use of a brain health mask ensures that the prediction corresponds to the brain age rather than chronological age, as discussed in Section~\ref{brain-age-estimation}.

In this project, the brain health is approximated by the absence of diagnoses related to mental health or nervous system disorders, defined by the \texttt{ICD10}-similarity metric (see Table~\ref{table:phenotype-features}).

\subsection{Population graph representation}
\label{section:population-graph-representation}

The population graph is stored in an extended \texttt{torch\_geometric.Data} object\footnote{\url{https://pytorch-geometric.readthedocs.io/en/latest/modules/data.html}}, with its most important fields listed in Table~\ref{table:population-graph}. The intermediate population graph representation has all its entries defined except for the feature vector \texttt{x} and the training, validation and test masks.

\setlength{\LTpost}{0pt}
\renewcommand{\arraystretch}{1.25}
% \begin{table}[]
%     \caption{The population graph data structure (excludes helper or utility fields).}\label{table:population-graph}
%     \centering
%     \begin{tabular}{lp{0.2\textwidth}p{0.5\textwidth}}
%         \hline
\begin{center}
\begin{longtable}[]{lp{0.175\textwidth}p{0.475\textwidth}}
    \caption{The population graph data structure (excludes helper or utility fields).}\label{table:population-graph}\\
    \hline \textbf{Field name} & \textbf{Type} & \textbf{Description} \\
    \hline
    \endfirsthead
    \multicolumn{3}{c}%
    {\tablename\ \thetable\ -- \textit{Continued from previous page}} \\
    \hline
    \textbf{Field name} & \textbf{Type} & \textbf{Description} \\
    \hline
    \endhead
    \hline \multicolumn{3}{r}{\textit{Continued on next page}} \\
    \endfoot
    \hline
    \endlastfoot
    % \texttt{num\_nodes} & long & Number of nodes (subjects) in the population graph. \\
    \texttt{subject\_index} & string array & UKB identifiers of the subjects. Stored in the same subject order as training masks, feature and label tensors; corresponds to the edge start and end values. \\
    \texttt{edge\_index} & $2\times 2|E|$ \hfill\newline long tensor & \texttt{edge\_index}$[0][i]=s_v$ and \hfill \newline \texttt{edge\_index}$[1][i]=s_w$ indicate a directed \hfill \newline edge $s_v \leadsto s_w$. Following the PyTorch Geometric API, to represent the undirected edge $(s_v, s_w) \in E$, another directed edge $s_w \leadsto s_v$ is added. \\
    \texttt{functional\_data} & dataframe & Row-indexed by subject with columns containing the flattened functional connectivity matrix entries. Empty if no functional data is used in the population graph. \\
    \texttt{structural\_data} & dictionary of \hfill \newline dataframes & Dictionary is indexed by the structural data modality, in this case cortical thickness, surface area, and grey matter volume. The corresponding dataframes are row-indexed by subject with columns containing the features of the relevant structural data modality. The dataframes are empty if no structural data is used. \\
    \texttt{quality\_control\_data} & dataframe & Row-indexed by subject with two columns containing Euler indices for the left and right hemispheres of the brain. Empty if no quality control data is used. \\
    \texttt{x} & $N \times F$ \hfill\newline float tensor & \textit{Unused at the intermediate stage.} Contains the full normalised feature vector (of $F$ features) for every graph node (subject). \\
    \texttt{y} & $N \times 1$ \hfill \newline float tensor & Contains the labels of training data, in this case chronological age. \\
    \texttt{brain\_health\_mask} & boolean array & \texttt{True} indicates that the subject has a healthy brain and can be used for training, and \texttt{False} otherwise. \\
    \texttt{train\_mask} & boolean tensor & \textit{Unused at the intermediate stage.} \texttt{True} if the subject belongs to the training set, and \texttt{False} otherwise. \\
    \texttt{validation\_mask} & boolean tensor & \textit{Unused at the intermediate stage.} \texttt{True} if the subject belongs to the validation set, and \texttt{False} otherwise. \\
    \texttt{test\_mask} & boolean tensor & \textit{Unused at the intermediate stage.} \texttt{True} if the subject belongs to the test set, and \texttt{False} otherwise.
    % \end{tabular}
\end{longtable}
\end{center}

\subsection{Testing}
As discussed in requirement R1 for this project (see Section~\ref{section:requirements-analysis}), it is important to ensure that the data is preprocessed and collected into the population graph correctly. To this end, both the UKB preprocessing (Section~\ref{section:ukb-preprocessing}) and the graph construction components have unit test modules asserting the correctness of the similarity matrices and example graph topologies.

\section{Graph transformation component}
\label{section:graph-transformation}

The graph transformation component is responsible for preparing the intermediate population graph representation for training by defining its normalised, concatenated feature vector as well as training, validation and test masks, using the parameters provided by the training component. The schematic diagram representing the population graph transformations in this component is shown in Figure~\ref{graph-transformation-component}.

\begin{figure}[h]
    \centering
    \includegraphics[width=\textwidth]{graph_transformation_component.pdf}
    \caption{\textit{Overview of the graph transformation component.}
    This component first sets the training, validation and test masks of the intermediate population graph. The masks are further used to reduce the dimensionality of fMRI data and normalise the remaining brain imaging data. The different brain imaging data modalities are combined into a single (node) feature tensor and assigned to the intermediate population graph (now masked with training sets), preparing it for training in the GNN training and evaluation component.
    }\label{graph-transformation-component}
\end{figure}

\subsection{Setting the training masks}
\label{setting-training-masks}
The first operation for preparing the population graph for training involves setting the \texttt{train\_mask}, \texttt{validation\_mask} and \texttt{test\_mask} fields of the intermediate population graph data structure (see Table~\ref{table:population-graph}). The masks are defined in the GNN training and evaluation component as this depends on the training procedure, not the structure of the population graph itself.

Following the brain age estimation method (Section~\ref{brain-age-estimation}), the masks are intersected with the \texttt{brain\_health\_mask}. Similarly to the function of the brain health mask, the three training masks are used to determine which nodes should be used to compute the loss function during various stages of training. For example, the model parameters are only updated based on the loss function for the subjects filtered with the \texttt{training\_mask}, and validation loss is computed only for the subjects filtered with the \texttt{validation\_mask}.

% TODO rb643: Again more for future implementation: I've been playing around with running PCA on the raw time-series as well as Diffusion Embedding on the connectivity matrices and they provide strikingly similar results. Thus is might be computationally efficient to use raw time-series PCA in the future. Generally only the first half dozen PC's are interesting anyway
\subsection{Functional connectivity matrix dimensionality reduction}
The flattened functional connectivity matrix results in over 70,000 features per patient. Since the population graph cannot be split into parts and during training must be kept entirely in memory, along with all model parameters, this quickly runs into memory issues when the entire dataset is used. To mitigate this issue, some techniques may be applied to reduce the dimensionality of functional data. This project uses principal component analysis (PCA) as the dimensionality reduction technique of choice as it is one the simplest to implement. When the functional data is used to construct the population graph, PCA transformation is fitted to the training set as specified by the \texttt{training\_mask}, and the same transformation is applied to the remaining subjects. This dimensionality reduction technique is optional in the pipeline, and it is possible to define how many principal components should be kept after the transformation to control the information loss. By default, only the most important 1\% of the components is kept so that the number of functional and structural features have the same order of magnitude.

\subsection{Structural MRI and quality control data normalisation}
Next, the raw features stored under \texttt{structural\_data} and \texttt{quality\_control\_data} are normalised to be in the range between $-1$ and $1$ with mean 0. Similar to the previous section, in order to avoid data leakage a standard scaler is fitted to the training set only (as specified by the \texttt{training\_mask}) and then applied to the validation and test sets. A separate transformation is applied to each structural data modality (cortical thickness, surface area, grey matter volume) and the quality control modality.

\subsection{Setting the transformed feature tensor}
The transformed functional, structural and quality control features are concatenated together into a single tensor, which is assigned to \texttt{x} in the population graph data structure (Table~\ref{table:population-graph}). This completes graph transformation for training.

Since the original neuroimaging dataframes and brain health masks are kept unchanged in the data structure, the same population graph can be prepared for, say, a different training fold by simply going through the transformation component again but with a different set of training masks, which will reset the values in the corresponding population graph fields.

\section{GNN architecture component}
\label{section:gnn-architecture}
% gcn_train(graph, device, n_conv_layers=0, layer_sizes=None, epochs=3500, lr=0.005, dropout_p=0, weight_decay=1e-5,
% log=True, early_stopping=True, patience=10, delta=0.005, cv=False, fold=0, run_name=None,
% min_epochs=1000):

The graph neural network component contains implementations for the graph neural network (GNN) architectures used in this project: the graph convolutional network (GCN, Section~\ref{training-gcn}) and the graph attention network (GAT, Section~\ref{training-gat}). The networks are implemented as two PyTorch modules called \texttt{BrainGCN} and \texttt{BrainGAT}, extending a shared \texttt{BrainGNN} module. Table~\ref{table:braingnn} presents the parameters used to define a \texttt{BrainGNN} instance. Then instantiating the subclasses \texttt{BrainGCN} and \texttt{BrainGAT} simply amounts to setting the \texttt{conv\_type} parameter to either \texttt{GCN} or \texttt{GAT}, and setting the convolutional layers to either \texttt{torch\_geometric.nn.GCNConv} or \texttt{torch\_geometric.nn.GATConv} respectively, reusing the rest of the code. The implementations for \texttt{GCNConv} and \texttt{GATConv} are available in the PyTorch Geometric library.


\begin{center}
    \begin{longtable}[]{p{0.275\textwidth}p{0.175\textwidth}p{0.475\textwidth}}
        \caption{The parameters for the \texttt{BrainGNN} module.}\label{table:braingnn}\\
        \hline \textbf{Parameter} & \textbf{Type} & \textbf{Description} \\
        \hline
        \endfirsthead
        \multicolumn{3}{c}%
        {\tablename\ \thetable\ -- \textit{Continued from previous page}} \\
        \hline
        \textbf{Parameter} & \textbf{Type} & \textbf{Description} \\
        \hline
        \endhead
        \hline \multicolumn{3}{r}{\textit{Continued on next page}} \\
        \endfoot
        \hline
        \endlastfoot
        
        \texttt{conv\_type} & string & Indicates the type of graph convolutional layer to be used (graph convolution or graph attentional layer). \\
        \texttt{layer\_sizes} & integer array & Lists the number of units in every hidden layer. The length of the array corresponds to the total number of hidden layers. \\
        \texttt{n\_conv\_layers} & integer & Number of convolutional layers. Must be in range $[0, \text{\texttt{len(layer\_sizes})}]$. The sizes of those layers are determined by the first \texttt{n\_conv\_layers} values of the \texttt{layer\_sizes} and \texttt{n\_node\_features} parameters. \\
        \texttt{num\_node\_features} & integer & Indicates the number of input features. \\ 
        \texttt{dropout\_p} & float $\in [0,1]$& The probability of ignoring the node in a hidden layer at training time. Used as a regularisation technique to reduce overfitting.
    \end{longtable}
    \end{center}

Listing~\ref{listing:braingnn} shows how the layers are combined in the \texttt{BrainGNN} architecture for a given set of parameters in Table~\ref{table:braingnn}. Hyperbolic tangent ($\tanh(\cdot)$) was chosen as the non-linearity (activation function) between each layer. This is because the population graph features and weights in neurons may be negative, in which case other popular activation functions such as $\mathrm{ReLU}(\cdot)$ may have an undesirable asymmetric response.

A \textit{dropout} (\texttt{torch.nn.Dropout}) layer was added between every fully connected layer in line 21 as a \textit{regularisation} technique to avoid overfitting. At each training step, a unit's weight is zeroed with probability \texttt{dropout\_p}, so that the neural network does not rely on any particular units when predicting age and learns more robust features.

\bigskip
\begin{code}
\caption{Simplified code snippet for \texttt{BrainGNN} instantiation and training.}
\label{listing:braingnn}
\medskip
\inputminted[frame=lines, linenos, breaklines=true, numberblanklines=false, style=colorful]{python}{code/brain_gnn_snippet.py}
\end{code}

% \begin{listing}
%     \caption{Simplified code snippet for \texttt{BrainGNN} instantiation and training.}
%     \label{listing:braingnn}
%     \medskip
%     \inputminted[frame=lines, linenos, breaklines=true, numberblanklines=false, style=colorful]{python}{code/brain_gnn_snippet.py}
%     \end{listing}

\section{GNN training and evaluation component}
\label{section:gnn-train-evaluate}

The main function of the GNN training and evaluation component is to select the best combination of population graph and GNN hyperparameters for each of the GCN and GAT architectures, and to report the results of the best model. It also contains the \textit{robustness evaluation} component, where robustness in this project is conceptualised as the predictive power drop when noise is added to the population graph nodes and/or edges. The schematic overview of the GNN training and evaluation component is shown in Figure~\ref{gnn-training-eval-component}.

\begin{figure}[h]
    \centering
    \includegraphics[width=\textwidth]{gnn_training_eval_component.pdf}
    \caption{\textit{Overview of the GNN training and evaluation component}.
    The component first generates sets of training, validation and test masks for a required number of folds, used to prepare the population graphs for training. Next, the model selection sub-component repeatedly generates new sets of hyperparameters, training various GNN and population graph combinations, and collecting their cross-validation performance scores on the validation set. As the hyperparameter tuning converges, the optimal GNN model is evaluated on the test set. Finally, the robustness measurement framework noisifies the nodes and edges of the optimal population graph, and evaluates the change in performance of the optimal GNN model.
    }\label{gnn-training-eval-component}
\end{figure}

\subsection{Model selection}
% Hyperparameter tuning, weights and biases

The brain age modelling task is challenging because of a vast combination of possible population graph and GNN hyperparameter choices. First, there is a choice of a neuroimaging modality combination for population graph node features (from functional, structural, and quality control data). Second, there are at least 32 combinations of possible non-imaging features used for similarity metrics (see Table~\ref{table:phenotype-features}), with each feature combination having a range of possible similarity thresholds. This does not account for the extension of using arbitrary linear combinations of non-imaging features for a similarity metric, and the fact that the UKB has over 760 more features that could potentially be included in similarity computation. Third, having fixed the population graph parameters, \texttt{BrainGNN} is in theory unlimited in its design and parameterisation possibilities (Table~\ref{table:braingnn}). Fourth, having fixed population graph and GNN hyperparameters, there are additional hyperparameters related to GNN training, such as the number of epochs and learning rate.

Having considered this, this section discusses further the motivation and reasoning behind most of the hyperparameter constraints and the training method. The full list of hyperparameter ranges that have been searched is included in Listing~\ref{listing:sweep-config}. 

\subsubsection{Memory constraints}
\label{section:memory}
% Tiago:  I won't get to the evaluation section today. It will be important to have in some part of your discussion the challenges of memory. That this is one big bottleneck in this type of analysis that needs to be tackled in the future. In case you don't mention this already

In the brain age estimation task with population graphs, the entire population graph containing node neuroimaging features, edges, GNN model training parameters and intermediate training states must be stored in memory at the same time. In addition, the population graph cannot be split into smaller independent batches since all nodes of the graph must be updated in a single training step, and it would be a non-trivial task to keep track of the edges and other interactions between the batches.

As a result, many hyperparameter combinations cause the model to run into memory issues. In practice, the models were limited to up to 8 gigabytes of GPU memory. This motivated the following hyperparameter constraints:

\begin{enumerate}
    \item \textit{Excluding functional data modality}. From preliminary experiments, using only functional and quality control data resulted in a poor graph neural network performance for the brain age estimation task. This is supported by the literature, such as the work of Niu et al.~\cite{niu2019improved}, which concluded that features derived from functional data were either not useful or even had a negative effect on predictive power, while structural features (particularly grey matter volume) were the most useful. Most importantly, functional data modality has over 70,000 features per subject (compared to 1,084 features for all remaining modalities combined), which results in an explosion in parameters for small estimated gain in performance. 
    
    Dimensionality reduction is possible, but from experience in the Data Science unit of assessment, cutting out half of the low-variance principal components can even worsen the predictive power and training time of the model instead of improving it. 
    
    While the functional data modality and dimensionality reduction technique were implemented~– indeed the goal of the pipeline was to support \textit{flexibility} to construct any population graph if needed (requirement R2, see Section~\ref{section:requirements-analysis}) – they were not used in training.
    \item \textit{Fixing the model size to a small set of shallow networks with a small number of units}. The learning rates were decreased and number of epochs increased to compensate for the smaller number of training parameters. The exact architecture is provided in Listing~\ref{listing:sweep-config}.
    \item \textit{Fixing the set of possible similarity feature sets and simlarity thresholds}. The default mode of averaging the feature-wise scores was used instead of arbitrary linear combinations, because, having no domain-specific knowledge of the relative importance of different confounders, this was the most non-informative, unbiased choice. From running some initial models, the feasible similarity thresholds have been all above 0.6, with even 0.7 or 0.8 often causing memory issues. This additionally motivated the use of as many non-imaging features as possible. \texttt{SEX} was always included in the feature sets because it is known to be an important confounder, significantly affecting the volume of the brain and sometimes even requiring a separate model for every sex~\cite{kaufmann2019}. The full list of population graphs tested can be found in Listing~\ref{listing:sweep-config}.
\end{enumerate}

\subsubsection{Hyperparameter tuning}
The hyperparameters were tuned using the Bayesian optimisation strategy provided by the \textit{Weights~\& Biases} (\texttt{wandb})~\cite{wandb} machine learning tracking and optimisation framework. Given the initial distributions of hyperparameters that should be searched (see Listing~\ref{listing:sweep-config}) and a command-line script that accepts hyperparameter combinations as arguments, \texttt{wandb} automatically selects the next set of hyperparameters that it believes is the most likely to improve a given performance metric. The choice of hyperparameters is based on both the initial hyperparameter distribution and the feedback from previously attempted values. For more information on Bayesian hyperparameter optimisation see Snoek et al.~\cite{snoek2012practical}.

Unlike in other hyperparameter tuning techniques such as grid search, Bayesian optimisation can use hyperparameter distributions defined over all real values in a given range. Consequently, hyperparameter search cannot be exhaustive and there is no obvious stopping point. In this project, hyperparameter tuning was run on each of the architectures until the models seemed to converge to similar performance with only marginal or no improvements as more hyperparameter combinations were tried. In general, this meant at least 100 hyperparameter combinations per GNN architecture, each hyperparameter combination trained five times for each cross-validation fold (see the next section), amounting to close to two weeks of total GPU compute time.

\subsubsection{Training procedure}
\label{section:training-procedure}

Before running the model selection and hyperparameter tuning procedure, the dataset is split into 90\% training/validation, and 10\% hold-out test set, stratifying by subject age and sex (as features that are the most likely to affect the brain age prediction). The test set is never used in training and hyperparameter tuning, and is only looked at during the evaluation procedure after the best hyperparameters for each of GCN and GAT have been selected (as indicated by the box at the bottom right in Figure~\ref{gnn-training-eval-component}). The subjects in the test set correspond to the \texttt{test\_mask} field of the population graph data structure as discussed in Table~\ref{table:population-graph} and Section~\ref{setting-training-masks}. While it is common to have multiple test sets (in techniques such as nested cross validation), this project does not use it because of the high computational cost and because the dataset is much larger than a typical brain imaging dataset (below 1,000 samples~\cite{parisot2018disease, cole2018brain}), making it more representative of the population.

The remaining data is used for hyperparameter tuning. Each hyperparameter combination is trained using a stratified five-fold \textit{cross-validation}: the training/validation dataset is split into five folds with 90\% training and 10\% validation subjects, again stratifying by age and sex. These correspond to \texttt{train\_mask} and \texttt{validation\_mask} of the population graph data structure. This model selection strategy to a large extent follows the one in Raschka~\cite{raschka2018model}, Section 3.7.

The summary of the above steps is shown in Figure~\ref{gnn-training-eval-component}: the data splits are visualised with a set of different training/validation/test masks at the top left corner of the GNN training and evaluation component (all of them having the same test mask but different training and validation masks). The masks of each fold are used to transform the graph (in the graph transformation component), and for each fold the prepared population graph is trained on a GNN implementation with a given set of hyperparameters (as suggested by the \texttt{wandb} framework). The optimal GNN model and population graph hyperparameters are selected based on which hyperparameter combination gives the smallest mean squared error (MSE), averaged over the validation sets of each fold. MSE was chosen as the standard loss function used in regression tasks.

To prevent overfitting and optimise the training procedure, for every fold the models are trained with \textit{early stopping}: while validation loss itself is not used to update the parameters, the training stops as validation loss begins to increase with decreasing training loss (indicating overfitting on training data). At this point the weights are reverted back to when the validation loss was the smallest. In this project, the models are trained for at least 1,000 epochs and stopped early if MSE does not decrease by at least 0.005 over 100 consecutive iterations. 

As the hyperparameter tuning converges, the most promising model for each of the \texttt{BrainGCN} and \texttt{BrainGAT} architectures is selected to be tested on the hold-out test set. The metrics most commonly used in the literature for evaluating performance for the brain age (or any other regression) task~\cite{pervaiz2020optimising, niu2019improved} are Pearson's correlation $r$ and coefficient of determination $r^2$ (which will be further discussed in Section~\ref{section:evaluation-metrics}).

\subsection{Robustness measurement}
\label{section:implementation-robustness}

The proposed \textit{extension} for the training and evaluation component was to evaluate the robustness of the GNN models to the noise in the population graph, measured by the rate of performance drop with increasing noise. In this project, robustness is evaluated in two ways, testing for two different effects of noise: one of them associated with the noisy neuroimaging data, and the other with the population graph representation of the dataset.

MRI data can be very noisy~\cite{pervaiz2020optimising,niu2019improved}, and most neuroimaging preprocessing pipelines (such as the ones used for UKB dataset) use various techniques to remove noise caused by subject head motion, image distortion and other factors~\cite{glasser2013minimal}. This does not eliminate all noise, however, which motivates the development of models that could correct for those errors. 
%Intuitively, since the GNN architectures account for the features and labels of the node's neighbourhood when predicting some response variable, incorporating features of less noisy neighbours could improve the prediction for a noisy node. 
To test how the model performance changes with noise, increasing proportions of nodes could be corrupted. One way to do this would be to add Gaussian noise, but this leaves it unclear how to choose the optimal noise amplitude. A more aggressive way would be to corrupt the nodes completely by, for example, permuting their features.

Another advantage of using the population graph (rather than treating every subject as an independent example) is that the associations between subjects (captured through graph edges and defined by subject similarities) incorporate additional non-imaging information which could improve the predictions and account for confounding effects. To test how much the model relies on edge information, some fraction of edges that should have been present in the population graph could be removed, observing the change in model performance. 


\section{Repository overview}
% The repository overview should be around one page in length and should describe the high-level structure of the source code found in your source code Repository; ... could be implemented as a table with folders/file names and the functionality implemented in those files

The repository follows the structure of the five key project components shown in Figure~\ref{pipeline-overview}. Its overview is presented in Table~\ref{table:repository-overview}.


\begin{center}
    \small
    \begin{longtable}[]{p{0.25\textwidth}p{0.3\textwidth}p{0.4\textwidth}}
        \caption{Repository overview.}\label{table:repository-overview}\\
        \hline \textbf{Component} & \textbf{Filename} & \textbf{Notes} \\
        \hline
        \endfirsthead
        \multicolumn{3}{c}%
        {\tablename\ \thetable\ -- \textit{Continued from previous page}} \\
        \hline
        \textbf{Component} & \textbf{Filename} & \textbf{Notes} \\
        \hline
        \endhead
        \hline \multicolumn{3}{r}{\textit{Continued on next page}} \\
        \endfoot
        \hline
        \endlastfoot
% \begin{table}[]
%     \small
%     \begin{tabular}{p{0.25\textwidth}p{0.3\textwidth}p{0.4\textwidth}}
%         \hline
%     \textbf{Component} & \textbf{Filename} & \textbf{Notes} \\  \hline
    UKB preprocessing 
            & \texttt{phenotype.py} & Contains enumeration of non-imaging features, mappings of non-imaging feature names to their UKB codes. \\
            & \texttt{ukb\_preprocess.py} & Functionality described in Section~\ref{section:ukb-preprocessing}. \\
            & \texttt{ukb\_preprocess\_test.py} &  Tests for similarity matrix code. \\ \hline
    Graph construction
            & \texttt{graph\_construct.py} &Functionality described in Section~\ref{section:graph-construction}.\\
            & \texttt{graph\_construct\_test.py} & Tests for example population graph correctness. \\ \hline
    Graph transformation
            & \texttt{graph\_transform.py} & Functionality described in Section~\ref{section:graph-transformation}.\\ \hline
            % & \texttt{ukb\_preprocess\_test.py} &  \\   
    GNN architectures
            & \texttt{brain\_gnn.py} & Implementation of \texttt{BrainGNN} and subclasses \texttt{BrainGAT}, \texttt{BrainGCN} (Section~\ref{section:gnn-architecture}).\\ \hline
            % & \texttt{ukb\_preprocess\_test.py} &  \\   
    GNN training
            & \texttt{brain\_gnn\_evaluate.py} & Population graph noisification and robustness measurement procedure (Section~\ref{section:gnn-train-evaluate}). \\
            & \texttt{brain\_gnn\_train.py} & Cross-validation training procedure (Section~\ref{section:gnn-train-evaluate}). \\
            & \texttt{wandb\_sweep.yaml} & Initial hyperparameter distributions used by \texttt{wandb} (Listing~\ref{listing:sweep-config}). \\
            & \texttt{wandb\_train.py} & Command line script used by \texttt{wandb} to automatically test new hyperparameters. \\
    % \end{tabular}
    % \end{table}
\end{longtable}
\end{center}

\chapter{Evaluation}
\label{chapter:evaluation}
% This is where Assessors will be looking for signs of success and for evidence of thorough and systematic evaluation as discussed in Section 8.3. Sample output, tables of timings and photographs of workstation screens, oscilloscope traces or circuit boards may be included. A graph that does not indicate confidence intervals will generally leave a professional scientist with a negative impression.
% As with code, voluminous examples of sample output are usually best left to appendices or omitted altogether.
% There are some obvious questions which this chapter will address. How many of the original goals were achieved? Were they proved to have been achieved? Did the program, hardware, or theory really work?
% Assessors are well aware that large programs will very likely include some residual bugs. It should always be possible to demonstrate that a program works in simple cases and it is instructive to demonstrate how close it is to working in a really ambitious case.

% ~2,000 words

\section{Model ranking and selection}
\label{section:model-ranking}
Following the hyperparameter tuning process described in Section~\ref{section:training-procedure}, the models were selected according to the following procedure (applied separately to the GCN and GAT model families):
\begin{enumerate}
    \item First, the models were ranked by ascending average MSE loss. The model with the lowest average MSE was chosen as the reference model.
    \item All models whose 1 standard deviation interval from their MSE did not overlap with the 1 standard deviation interval of the reference model MSE were excluded from ranking.
\end{enumerate}

The cross-validation performance of the best-scoring models selected by the above procedure is shown in Figure~\ref{figure:gat-gcn-rank}. The hyperparameters for each of the short-listed models are listed in Appendix~?.

% \begin{figure}[]
%     \centering
%     \includegraphics[width=\textwidth]{gcn_model_selection.pdf}
%     \caption{Highest scoring population graph and GCN model parameter combinations.}\label{figure:gcn-rank}
% \end{figure}

% \begin{figure}[]
%     \centering
%     \includegraphics[width=\textwidth]{gat_model_selection.pdf}
%     \caption{Highest scoring population graph and GAT model parameter combinations.}\label{figure:gat-rank}
% \end{figure}

\begin{figure}[h]
    \centering
    \includegraphics[width=\textwidth]{model_selection.pdf}
    \caption{Highest scoring population graph and GNN model parameter combinations.}\label{figure:gat-gcn-rank}
\end{figure}

Although both best-ranked (reference) models (GCN1 and GAT1) have relatively high variance, they still seem to be the most promising and therefore have been selected for further evaluation. Their population graph specification and GNN architecture hyperparameters are listed in Table~\ref{table:best-hyperparameters}.

\begin{table}[]
    \caption{Best performing population graph and GNN model parameter combinations during the model selection process.}\label{table:best-hyperparameters}
    \centering
    \small
    \begin{tabular}{p{0.3\textwidth}p{0.3\textwidth}p{0.3\textwidth}}
        \hline
    \textbf{Hyperparameter} & \textbf{GCN1} & \textbf{GAT1} \\  \hline
        Similarity feature set & \texttt{FI}, \texttt{FTE}, \texttt{ICD10}, \texttt{MEM}, \texttt{SEX} & \texttt{FI}, \texttt{ICD10}, \texttt{MEM}, \texttt{SEX} \\
        Similarity threshold & 0.9 & 0.8 \\ \hline
        Layer sizes & [1024, 512, 512, 256, 256, 1] & [2048, 1024, 512, 256, 128, 1] \\
        \# convolutional layers & 5 & 2 \\
        Dropout & $3.22 \times 10^{-1}$ & $3.14 \times 10^{-3}$ \\
        Learning rate & $6.98 \times 10^{-3}$ & $1.34 \times 10^{-2}$ \\
        Weight decay & $1.31 \times 10^{-2}$ & $6.05 \times 10^{-4}$ \\ \hline
\end{tabular}
\end{table}

\section{Evaluation metrics}
\label{section:evaluation-metrics}
The main performance metrics used for most regression problems, including brain age estimation task, are \textit{Pearson's correlation} and \textit{coefficient of determination}.

\subsubsection{Pearson's correlation}
For sets of true labels $\mathbf{y}  = [y_1 \dots y_N]$ with mean $\bar{y}$ and predicted labels $\mathbf{\hat{y}} = [\hat{y}_1 \dots \hat{y}_N]$, \textit{Pearson's correlation} is computed as

\begin{equation}
    r(\mathbf{y}, \mathbf{\hat{y}}) = \frac{\mathrm{cov}(\mathbf{y}, \mathbf{\hat{y}})}{\sigma_{\mathbf{y}} \sigma_{\mathbf{\hat{y}}}},
\end{equation}

where $\mathrm{cov}(\cdot, \cdot)$ denotes covariance and $\sigma$ stands for standard deviation. 

\subsubsection{Coefficient of determination}
The \textit{coefficient of determination} indicates how much variance in the features $\mathbf{X}$ could be explained by the model. It is computed as 
\begin{equation}
    r^2 = 1 - \frac{\sum_{i} (y_i - \bar{y})^2}{\sum_{i} (y_i - \hat{y}_i)^2}.
\end{equation}

Higher values for both metrics (with maximum 1) indicate a higher level of agreement between the true and predicted labels and therefore higher predictive power.


\section{Test set performance of selected models}
Generally in literature, after using cross-validation for model selection, the model is retrained on the entire dataset before giving a point estimate on a hold-out test set~\cite{raschka2018model}. This is because training on more data, especially when the dataset is small, allows to learn more patterns and therefore give better predictions on the unseen data. However, in this project the validation set was also used for early stopping since neural networks are especially prone to overfitting~\cite{prechelt1998automatic}. Some investigation of the hyperparameter tuning has shown that applying the stopping criteria discussed in Section~\ref{section:training-procedure} on just the training set would have still led to convergence only after the model has already overfit on the unseen validation labels. On the other hand, it is unclear how the stopping criteria should be adjusted when the training set size increases. 

Considering that the UKB dataset is large and that retraining the model with more data but without early stopping might not pay off for the loss in generalisation, all cross-validation folds were kept for test set performance estimation. Table~\ref{table:test-performance} gives the hold-out test set estimates for the metrics discussed in Section~\ref{section:evaluation-metrics}.

\begin{table}[h]
    \caption{Average test set performance of GCN1 and GAT1 models. \\ The results are within one standard deviation.}\label{table:test-performance}
    \centering
    \small
    \begin{tabular}{ccc}
        \hline
    \textbf{Model} & $r$ & $r^2$ \\  \hline
        GCN1 & $0.675 \pm 0.008$ & $0.445 \pm 0.010$ \\
        GAT1 & $0.670 \pm 0.005$ & $0.477 \pm 0.008$ \\ \hline
\end{tabular}
\end{table}


\section{Permutation testing}

\section{Population graph robustness to noisy features}
A desirable property for the real-world machine learning models is their robustness, defined as tolerance to the noise and inconsistency in data.
For example, as discussed in Section~\ref{section:implementation-robustness}, it is useful if the model can retain its performance even when the data contains particularly noisy MRI scans. Population graphs trained on graph neural networks could do this by exploiting the neighbourhoods that are hopefully less noisy than a given node. Whether they are actually capable of doing so could be tested by either adding noise to an increasing proportion of nodes or depriving the nodes from their neighbourhoods. 

For the first type of robustness evaluation, the normalised feature tensor \texttt{x} of the population graph (see Table~\ref{table:population-graph}) is perturbed with the additive white Gaussian noise of variance 0.5 (as the normalised features are between $-1$ and 1) for the increasing proportion of nodes – the levels chosen were 1\%, 5\%, 10\%, 20\%, 30\% and 50\%. Following the renormalisation of the tensor, the model is retrained again on the training dataset and tested on the hold-out test set, measuring the change in performance. To make sure that any effect on the evaluation metrics is due to added noise and not the changing dataset split, the model is trained on a single dataset split while the noise is added to different subjects. For each of the GCN and GAT models, this results in five training procedures at each noise level.


\section{Population graph robustness to loss of structure}
The the purpose of the second type of robustness is to explore whether the population graph structure itself is useful in estimating the brain age for a particular node. To this end, instead of adding noise to the nodes, an increasing proportion of edges is removed from the population graph. Similarly to the previous section, the selected proportions of edges removed were 1\%, 5\%, 10\%, 20\%, 30\% and 50\%. For each of these levels the training procedure is carried out five times using a different random seed. The more edges are removed, the less neighbourhood structure the graph neural network models can exploit, having to rely on the node features only. 





% \section{Comparison against existing benchmarks}

% Compare to the Kaufmann et al.'s \textit{xgboost} approach \cite{kaufmann2019} ($r \sim 0.93$); and the other package that was cited in the same paper.
% Possibly compare to other non-graph (relatively baseline) (neural network) architectures, e.g. ElasticNet, MLP,...



\chapter{Conclusion}

% This chapter is likely to be very short and it may well refer back to the Introduction. It might properly explain how you would have planned the project if starting again with the benefit of hindsight.

%  ~500 words

\section{Successes and failures}

The project has achieved all of its success criteria and requirements (Section~\ref{section:requirements-analysis}, Appendix~\ref{chapter:project-proposal} Project Proposal), representing the UKB dataset as a population graph, implementing the two GNN frameworks, and evaluating the results. It has also made a lot of progress on its \textit{extensions}, measuring the robustness of the GNN models and increasing the flexibility of the preprocessing pipeline to more preprocessing options.

\section{The main lesson}
This project has been exploring two things – the brain age estimation problem, and graph neural networks as a way of solving it. Both of those fields are currently at the height of their ongoing research, with some of the main references that this project heavily relied on being only a few months old~\cite{kaufmann2019, niu2019improved, pervaiz2020optimising}.

While I am very happy for having used this opportunity to \textit{learn} about the cutting-edge approaches in machine learning research and \textit{get hands-on experience} with the tools for implementing them, this project has taught me first-hand an important lesson in approaching predictive analysis problems: \textit{when developing models in practice, always try the simplest approaches first, and only then, if the model does not perform well, build up to sophisticated (``deep'') frameworks}. Both sophisticated graph neural network architectures were not even close to much simpler frameworks presented in literature. For this reason I am particularly proud of coming up with what I called the robustness measurement framework – it insightfully showed that the graph representation was not only not very useful for this problem (as removing edge information had no effect on model performance), but could even have a negative effect when the data is noisy.

% Mention Niu et al. 2019 raising the issue that there is systematic bias in brain age gap prediction but not many studies use this knowledge to correct for it. 

\section{Towards a more powerful preprocessing pipeline}
The main success of this project was therefore its step towards a preprocessing pipeline, which could process several brain imaging and non-imaging modalities at once into a unified and consistent manner. This is important in neuroimaging research regardless of the downstream task or analysis method: even for the same dataset the preprocessing workflow alone could change the results~\cite{salehi2020there}, while at the same time the workflows vary significantly between different research teams. For example, in a study by Botvinik-Nezer et al.~\cite{botvinik2019variability} none of the 70 teams chose the same workflow for an identical problem. Perhaps a different preprocessing pipeline would improve the results even for the brain age estimation task in this project.

The preprocessing framework is also a project that could especially benefit from good software engineering and computer science skills in order to design a general that is efficient, flexible and simple to apply to different contexts.
\begin{itemize}
    \item While it was not practical to do this extension for the UKB dataset that uses its own organisation, the preprocessing pipeline could be adapted to work with raw neuroimaging files, such as the standard BIDS\footnote{\urlstyle{https://bids.neuroimaging.io}} format, extending the framework to work with a many more neuroimaging datasets and parcellations (that would now be a part of the preprocessing framework).
    \item Including more powerful functional imaging preprocessing and dimensionality reduction steps.
    \item Including the option to use genetic data. DNA methylation and other genetic data is widely used in other neuroimaging studies, improving predictive power of the models \cite{cole2018brain,parisot2018disease}.
\end{itemize}



%TC:ignore
%%%%%%%%%%%%%%%%%%%%%%%%%%%%%%%%%%%%%%%%%%%%%%%%%%%%%%%%%%%%%%%%%%%%%
\addcontentsline{toc}{chapter}{Bibliography}
\printbibliography

%%%%%%%%%%%%%%%%%%%%%%%%%%%%%%%%%%%%%%%%%%%%%%%%%%%%%%%%%%%%%%%%%%%%%

\appendix
% Assessors like to see some sample code or example circuit diagrams, and appendices are the sensible places to include such items. Accordingly, software and hardware projects should incorporate appropriate appendices. Note that the 12,000 word limit does not include material in the appendices, but only in extremely unusual circumstances may appendices exceed 10–15 pages – if you feel that such unusual circumstances might apply to you you should ask your Director of Studies and Supervisor to apply to the Chairman of Examiners. It is quite in order to have no appendices. Appendices should appear between the bibliography and the project proposal.


\chapter{Hyperparameters of shortlisted models}
\label{appendix:hyperparameters}

This Appendix lists the full hyperparameter lists of the best performing models as selected in Evaluation procedure (Chapter~\ref{chapter:evaluation}, Section~\ref{section:model-ranking}). The Tables use encodings given in Tables~\ref{table:sf-encoding} and~\ref{table:ls-encoding}.

\begin{table}[h]
    \caption{Similarity feature set encoding.}\label{table:sf-encoding}
    \centering
    \small
    \begin{tabular}{cccccc}
        \hline
    \textbf{Encoding} & \texttt{FI} &  \texttt{FTE}& \texttt{ICD10}& \texttt{MEM}& \texttt{SEX}\\  \hline
        SF1 & Yes & Yes & Yes & Yes & Yes \\
        SF2 & Yes & No & Yes & Yes & Yes \\
        SF3 & No & Yes & Yes & Yes & Yes \\
        SF4 & Yes & Yes & No & Yes & Yes \\ \hline
\end{tabular}
\end{table}

\begin{table}[h]
    \caption{Layer size encoding.}\label{table:ls-encoding}
    \centering
    \small
    \begin{tabular}{cccccc}
        \hline
    \textbf{Encoding} & \textbf{Layer sizes} \\  \hline
        LS1 & [1024, 512, 512, 256, 256, 1] \\ 
        LS2 & [1024, 512, 512, 512, 256, 256, 1] \\
        LS3 & [1024, 512, 256, 128, 128, 1] \\
        LS4 & [2048, 1024, 512, 256, 128, 1] \\
        LS5 & [512, 512, 512, 256, 128, 1] \\ \hline
\end{tabular}
\end{table}

\begin{sidewaystable}[h!]
    % \begin{table}[]
    \caption{Shortlisted population graph and GCN model parameter combinations during the model selection process.}\label{table:shortlisted-gcn}
    \centering
    \centering
    \small
    \begin{tabular}{lccccccc}
        \hline
    \textbf{Hyperparameter} & \textbf{GCN1} & \textbf{GCN2} & \textbf{GCN3} & \textbf{GCN4} & \textbf{GCN5} & \textbf{GCN6} & \textbf{GCN9} \\  \hline
        Similarity feature set &  SF1 & SF3 & SF3 & SF2 & SF2 & SF2 & SF2 \\
        Similarity threshold & 0.9 & 0.8 & 0.8 & 0.8 & 0.8 & 0.8 & 0.8\\ \hline
        Layer sizes & LS1 &  LS2 & LS3 & LS5 & LS3 & LS3 & LS4 \\ 
        \# convolutional layers & 5 & 3& 1& 2& 5& 3& 4\\ 
        Dropout &  0.321941 & 0.042080& 0.048596& 0.237940& 0.375442& 0.386998& 0.426491\\ 
        Learning rate & 0.006984& 0.006187& 0.005095& 0.004731& 0.015796& 0.010273& 0.003504\\ 
        Weight decay & 0.013118& 0.002084& 0.016171& 0.002517& 0.003114& 0.005341& 0.018943\\ \hline
\end{tabular}
 
\bigskip\bigskip

    \caption{Shortlisted population graph and GAT model parameter combinations during the model selection process.}\label{table:shortlisted-gat}
    \centering
    \small
    \begin{tabular}{lccccccccc}
        \hline
    \textbf{Hyperparameter} & \textbf{GAT1} & \textbf{GAT2} & \textbf{GAT3} & \textbf{GAT4} & \textbf{GAT5} & \textbf{GAT6} & \textbf{GAT7} & \textbf{GAT8} & \textbf{GAT9} \\  \hline
    Similarity feature set & SF2 & SF1 & SF2& SF1 & SF1& SF1& SF1& SF2& SF2\\
    Similarity threshold & 0.8 & 0.9& 0.8& 0.9& 0.9& 0.9& 0.9& 0.8& 0.8\\ \hline
    Layer sizes& LS4& LS3& LS3& LS5& LS3& LS3& LS3& LS5& LS5\\
    \# convolutional layers & 2& 2& 3& 2& 3& 2& 3& 3& 3\\
    Dropout & 0.003142& 0.306806& 0.104624& 0.327091& 0.407471& 0.323481& 0.291117& 0.455777& 0.381829\\
    Learning rate& 0.013365& 0.001679& 0.003412& 0.002482& 0.003246& 0.001462& 0.006769& 0.006813& 0.003820\\
    Weight decay& 0.000605& 0.002071& 0.036676& 0.001549& 0.006715& 0.002475& 0.000844& 0.001483& 0.003226\\ \hline
\end{tabular}
    \end{sidewaystable}

\begin{refsection}
\documentclass[12pt,a4paper,twoside, hidelinks]{article}
\usepackage{bookmark}
\usepackage{amsmath}
\usepackage{parskip}
\usepackage{enumitem}
\usepackage{hyperref}
\urlstyle{same}
\usepackage{xcolor}
\usepackage[multiple]{footmisc}
\usepackage[margin=25mm]{geometry}
% \usepackage[backend=biber, maxnames=4]{biblatex}
% \addbibresource{stankeviciute-proposal.bib}

\begin{document}

\begin{center}
\Large
Computer Science Tripos -- Part II -- Project Proposal\\[4mm]
\LARGE
Graph neural networks for age prediction from neuroimaging data \\[4mm]

\large
Kamilė Stankevičiūtė (\texttt{ks830}), Gonville \& Caius College

\today % October 2019
\end{center}

\vspace{5mm}
\textbf{Project Originator:} Tiago Azevedo

\textbf{Project Supervisors:} Tiago Azevedo, Alexander Campbell, Prof Pietro Liò

\textbf{Directors of Studies:} Dr Timothy~M.~Jones, Dr Graham~Titmus

\textbf{Project Overseers:} Prof Jon~Crowcroft, Dr Thomas~Sauerwald

% Main document

\section*{Introduction}
% The problem to be addressed.

% [Tiago] Why NNs and not something else? You probably want one sentence of motivation saying they have been very successful in other fields, and then one sentence that as a consequence they might help physicians.
% \textit{...Neural networks provide the opportunity to capture the similarities between patients and trends which might help physicians to understand the mechanisms of the disease and in turn find more effective treatments...}

A graph neural network (GNN) is a type of a neural network that operates on graph inputs and is used for tasks like node classification, link prediction and clustering (geometric deep learning). GNNs have recently become popular and proved successful in a broad range of real-world applications, such as text and image classification, knowledge graphs, and interaction modelling in physical and biological systems. \cite{zhou2018gnn}

One domain where graphs offer a natural representation is social networks and \textit{populations}, with nodes representing individuals (their features and labels), and edges corresponding to associations between individuals according to some heuristic or a formally defined similarity metric. The reason why such graph representation is considered to be useful in the geometric deep learning context is that the network can make use of both the individual features (node feature vectors) and the overall trends in the population through pairwise similarities (graph edges), \cite{parisot2017spectral} inferring the label of an individual node both from the node itself and from its neighbourhood.
% The graph structure is also helpful when incorporating multiple modalities of data, which is often the case for medical records containing, for example, imaging and non-imaging data. 

\section*{Project description}
This project was inspired by Parisot et al.'s \cite{parisot2017spectral, parisot2018disease} state-of-the-art application of a type of a GNN called Graph Convolutional Network (GCN) to the population graphs of healthy controls and patients with neurological or neurodegenerative disorders. In these papers, the GCN (adapted from Kipf and Welling \cite{kipf2017semi}) was used in a semi-supervised manner for two classification tasks: 1) prediction of autism spectrum disorder (ASD) from the ABIDE dataset and 2) prediction of a progressive form of Mild Cognitive Impairment (MCI) that develops into Alzheimer's disease (AD) from the ADNI dataset.

Moreover, a recent paper by Kaufmann et al. \cite{kaufmann2019} has linked the incidence of common brain disorders, including ASD, MCI, and AD as well as others, to the deviation between chronological and biological brain ageing. These results suggest that being able to estimate the subject's age from the neuroimaging data may be important in understanding the mechanisms of those disorders and helping physicians to find more effective treatments.

The aim of this project will therefore be to adapt the population graph approach of Parisot et al. \cite{parisot2017spectral, parisot2018disease} to a regression task on the UK Biobank dataset, predicting the subject's age based on neuroimaging data, and comparing it to another successful geometric deep learning architecture such as the Graph Attention Network. \cite{velickovic2018graph} The performance of the networks will be evaluated on the standard metrics, e.g. the coefficient of determination~$r^2$.

\section*{Starting point}
% Describe existing state of the art, previous work in this area,
%   libraries and databases to be used. Describe the state of any
%   existing codebase that is to be built on.

The source code for the implementation of Kipf and Welling's \cite{kipf2017semi} GCNs and Parisot et al.'s \cite{parisot2017spectral, parisot2018disease} first classification task is publicly available online.\footnote{\url{https://github.com/tkipf/gcn}}\footnote{\url{https://github.com/parisots/population-gcn}}

I will be using PyTorch for this project because of its support for machine learning on structured graph data. In particular, PyTorch Geometric (PyG)\footnote{\url{https://github.com/rusty1s/pytorch_geometric}} – a geometric deep learning extension library – will make the implementation, iteration and extensions to the model more flexible in addition to performance improvements and simplified APIs.
% making the final library more accessible and extensible, contributing to the open-source community

I have experience with the basics of TensorFlow\footnote{Five-course Deep Learning specialisation by deeplearning.ai on Coursera}\footnote{Google's Machine Learning Crash Course and follow-up courses.} and no experience with PyTorch or graph neural networks. I have attended or will study (possibly in advance) the CST courses related to the subject of this project, such as IA Machine Learning and Real-World Data, IB Artificial Intelligence, II Data Science, II Bioinformatics, and II Machine Learning and Bayesian Inference.

\subsection*{Dataset}

I will be using the data from the UK Biobank, kindly preprocessed and provided by Dr~Richard Bethlehem of the Department of Psychiatry.

The UK Biobank is a large dataset containing comprehensive phenotypic, genetic, MRI and other data from the total of over 500,000 participants.\footnote{\url{https://www.ukbiobank.ac.uk/participants/}} In this project, I will be using the subset of this dataset with only those subjects who had the neuroimaging data collected and preprocessed (approximately 20,000 participants). This includes both structural (T1, T2 FLAIR) and functional (resting state fMRI) data, preprocessed with the standard UK Biobank pipelines\footnote{\url{https://biobank.ctsu.ox.ac.uk/crystal/crystal/docs/brain_mri.pdf}} and additionally denoised and parcellated (in several common parcellations) by Dr Bethlehem. 
% I am likely to be using the correlation matrices and raw parcellated time series for functional and features like coritical thickness for structural data.
\newpage
\section*{Work to be done}
\label{section:work}

% [Tiago] bullet points should start with the same sentence structure

% Describe the technical work.
The following are lists of explicit deliverables to be implemented.

\textbf{Graph neural network framework}
\begin{enumerate}[label=G\arabic*.]
  \item The data is preprocessed into features and is ready for analysis. % [Tiago] "data cleaned and is ready for analysis" 
  \item Definition of the similarity metric to be used in connecting the graph. The graph is connected based on that similarity metric to be processed by graph neural networks.
  \item Implementation of Kipf's GCN \cite{kipf2017semi} for the age regression task.
  % [Tiago] GAT originally doesn't allow for weighted edges. You probably want to say GAT because of interesting results in previous literature. Thus, you can probably divide this point: (1) implementation of another graph NN layer, (2) Include weights (in theory you can even edit the message passing mechanism in GCN to multiply by the weights, just like you are suggesting for GAT) 
  \item Implementation of the Graph Attention Network for comparing its performance to the Graph Convolutional Network.
  % [Tiago] What exactly would you be testing points? Eg. what a unit test would consist of?
  % [Tiago] I just recalled that one thing we discussed could be how it handles missing data (eg. a certain percentage without some data), which could create an interesting view on robustness and semi-supervised learning. Maybe this could go to extension (or "personal" extension in case you have time and you can say you had one more extension than initially planned)
\end{enumerate}

\textbf{Evaluation framework}

\begin{enumerate}[label=E\arabic*.]
  \item  Comparison of the alternative graph neural network models using the coefficient of determination $r^2$.
\end{enumerate}

\section*{Success criteria}
% Describe what you expect to be able to demonstrate at the
% end of the project and how you are going to evaluate your achievement.
The project will be successful if the following items will have been implemented.
\begin{enumerate}[label=SC\arabic*.]
  \item Representation of the UK Biobank data as a population graph with nodes representing the individuals and edges representing associations between them based on pairwise similarity.
  \item The Graph Convolutional Network for age regression on the population graph.
  \item The Graph Attention Network for the same task.
  \item The evaluation framework for comparing the performance of the two graph neural networks.
\end{enumerate}

\section*{Evaluation of the project}
The performance of the graph neural networks will be measured across several metrics. The main metric to evaluate a regression task, in contrast the classification in Parisot et al. \cite{parisot2018disease}, is the coefficient of determination~$r^2$. 

\section*{Possible extensions}
% Potential further envisaged evaluation metrics or extensions.
\begin{enumerate}[label=PE\arabic*.]
  \item An additional metric that could be used to evaluate the performance of the networks is \textit{robustness} to missing or noisy data. Robustness, which could be defined as \textit{the rate at which the predictive power drops as more information is removed from the nodes}, would reveal how important is the neighbourhood (edge) information for accurate predictions compared to the node features only.
  \item Implement spectral filter computation with \textit{Cayley polynomials} instead of using Chebyshev polynomials. Cayley polynomials have been introduced in a paper by Levie et al. \cite{levie2017cayleynets} and were mentioned in Parisot et al. \cite{parisot2018disease} as a possible improvement.
  \item The main implementation of the graph neural network relies on manually handcrafted features from preprocessed brain imaging data. Time permitting, an extension could be to create a package that can be used after any standard neuroimaging preprocessing pipeline (e.g. with results in BIDS\footnote{\url{https://bids.neuroimaging.io}} format) to extract these features, and possibly improve upon as well as create new ones. This would make execution of the model more efficient, robust and generalisable.
  \item Implement weighted edges in the Graph Convolutional Network and Graph Attention Network, as the main implementations will have binary edges.
  % \item Implement a \textit{custom similarity metric}. The metrics used in the work by Parisot et al. \cite{parisot2018disease} were defined arbitrarily by the authors based on very few features. Learning a different similarity metric based on more combinations of features could possibly result in a better performance of the classifier.

\end{enumerate}


\section*{Timetable and milestones}
\label{section:timetable}

% A work plan of perhaps ten or so two-week work-packages,
% as well as milestones to be achieved along the way. Provide a
% target date for each milestone.

% [Tiago] you can specify which parts of the work you intend to implement in each 2-week time frame. This will help you having a better idea of how you are keeping up/behind.

%  (01/10/2019 – 16/10/2019)
\textbf{Michaelmas weeks 0–1}
\begin{itemize}
  \item Work on project proposal.
\end{itemize}

\textbf{Milestones.} Submit Phase 1 report by 14/10/2019. Submit draft proposal by 18/10/2019.

% (17/10/2019 – 06/11/2019)
\textbf{Michaelmas weeks 2–4}
\begin{itemize}
  \item Get access to the UK Biobank data and get familiar with its features.
  \item Define a possible graph similarity metric.
\end{itemize}

\textbf{Milestones.} Submit final project proposal by 25/10/2019.

% (07/11/2019 – 20/11/2019)
\textbf{Michaelmas weeks 5–6}
\begin{itemize}
  \item Write code for connecting the nodes based on a similarity metric.
  \item Connect the nodes (with their features) into a graph.
  \item Start working on the implementation of the Graph Convolutional Network (e.g. define loss and random label removal for semi-supervised training, start implementing the layers).
\end{itemize}

% (21/11/2019 – 04/12/2019)
\textbf{Michaelmas weeks 7–8} 
\begin{itemize}
  \item Work on the implementation of layers for the Graph Convolutional Network. Compute the general performance metrics.
  \item Start working on Graph Attention Network implementation for the same task.
\end{itemize}

\textbf{Michaelmas vacation}
\begin{itemize}
  \item Continue working on and finish the neural network implementations, compute performance metrics.
  \item Work on graph neural network evaluation: implement the robustness measurement framework.
  \item Measure the robustness of the neural networks.
  \item Start writing the dissertation and the project progress report.
\end{itemize}

\textbf{Milestones.} Complete the implementation of the main part of the project.

% (16/01/2020 – 29/01/2020)
\textbf{Lent weeks 0–2}
\begin{itemize}
  \item Finish the progress report, prepare for the presentation.
  \item Implement Cayley polynomials.
  \item Start working on the data preprocessing pipeline.
\end{itemize}
 
\textbf{Milestones.} Submit progress report by 31/01/2020.

% (30/01/2020 – 19/02/2020)
\textbf{Lent weeks 3–5}
\begin{itemize}
  \item Continue implementing the data preprocessing pipeline.
  \item Start working on the implementation of weighted edges.
\end{itemize}


% (20/02/2020 – 11/03/2020)
\textbf{Lent weeks 6–8}
\begin{itemize}
  \item Finish implementing the data preprocessing pipeline.
  \item Finish implementing the weighted edges.
  \item Continue working on the dissertation write-up.
\end{itemize}

\textbf{Easter vacation}
\begin{itemize}
  \item Complete the dissertation draft and send it for review.
  \item Edit the draft based on the feedback received.
\end{itemize}

\textbf{Milestones.} Send out the complete draft for review by 27/03/2020. Submit dissertation early by 20/04/2020.

% (24/04/2020 – 06/05/2020)
\textbf{Easter weeks 0–2}

 Time reserved for any unexpected issues.

 \section*{Resource declaration}

 For this project I will be using my personal MacBook Pro (2019, with 1.4 GHz Quad-Core Intel Core i5 processor and 8GB of RAM). Training the model will require the use of GPUs provided by the Computational Biology Group (as confirmed by Prof Pietro Liò). To prevent any loss of data, both the source code and the \LaTeX\ source will be stored on my machine, private GitHub repositories, and Google Drive, as well as regularly backed up on an external HDD.

 \newpage
% \medskip 
% \printbibliography
\bibliographystyle{unsrt}
\bibliography{references}

\end{document}
\printbibliography[title=References]
\end{refsection}
%TC:endignore

\end{document}
