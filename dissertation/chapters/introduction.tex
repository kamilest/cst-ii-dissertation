\chapter{Introduction}
\pagenumbering{arabic}

% The Introduction should explain the principal motivation for the project. Show how the work fits into the broad area of surrounding Computer Science and give a brief survey of previous related work. It should generally be unnecessary to quote at length from technical papers or textbooks. If a simple bibliographic reference is insufficient, consign any lengthy quotation to an appendix.

% ~500 words

Many common neurological and neurodegenerative disorders, such as Alzheimer’s disease, schizophrenia and multiple sclerosis, have been associated with abnormal patterns of apparent ageing of the brain~\cite{kaufmann2019}. This link has inspired numerous studies in \textit{brain age estimation} using neuroimaging data~\cite{franke2019ten}. The \textit{brain age gap}, defined as the discrepancy between the estimated brain age and the true chronological age, is a powerful biomarker not only for understanding the biological pathways behind the ageing process, but also for assessing an individual’s risk to various brain disorders and identifying new personalised treatment strategies~\cite{tuncc2019deviation}.


Numerous studies exist applying machine learning algorithms to the problem of brain age estimation, typically using structural magnetic resonance imaging (MRI) and genetic data~\cite{franke2019ten}. They tend to predict the brain age for each sample independently from the others, modelling healthy controls separately from individuals with brain disorders, and often developing separate models for each sex~\cite{kaufmann2019,niu2019improved}. They tend not to explicitly consider patterns that are shared across the subgroups of subjects, training on the full dataset but controlling for confounding effects. Moreover, they rarely include other important brain imaging modalities such as functional MRI (fMRI) time-series data, or clinical expertise of neurologists and psychiatrists, even though a combination of different modalities has been shown to improve the results~\cite{niu2019improved}. 


This dissertation proposes a pipeline that can flexibly combine the richness of minimally preprocessed neuroimaging as well as non-imaging modalities to predict the brain age in a clinically relevant fashion. The data is combined using the \textit{population graph} representation of patients, where the nodes of the population graph contain subject-specific neuroimaging data, and edges represent pairwise subject similarities using non-imaging data. In addition to controlling for confounding effects, these similarities exploit the neighbourhood information when predicting node labels – an approach which has successfully been applied to a variety of problems in both medical and non-medical domains~\cite{parisot2018disease}. 


The population graph is used as input to two types of neural networks that can operate on graph-structured data – the \textit{graph convolutional}~\cite{kipf2017semi} and the \textit{graph attention}~\cite{velickovic2018graph} networks – to predict the brain age. This follows closely the approach presented in the work of Parisot et al.~\cite{parisot2017spectral,parisot2018disease}, where population graphs are used to classify patients as healthy or suffering from some brain disorder (i.e. autism spectrum disorder and Alzheimer's disease).


Unlike in other state-of-the-art models, graph neural networks trained on population graphs have the potential to learn from the entire cohort of healthy and affected subjects of both sexes at once, capturing a wide range of confounding effects~\cite{ruigrok2014meta, lancet2016sex} and detecting the variation in brain age trends of different sub-populations of subjects. 

% \section{Related work}
% \subsection{Population graphs}
% Semisupervised population graphs in Parisot et al. for classifying healthy patients and patients having Mild Cognitive Impairment/Alzheimer's or autism spectrum disorder. These achieve state-of-the-art accuracies around 70\% even for a binary classification task which indicates that brain conditions are generally a complex task. Due to the different nature of the task and different evaluation metrics, this performance cannot be directly compared to the predictive power of the brain age regression task.

% \subsection{Graph neural networks}

% \subsection{Brain age prediction from neuroimaging data}
% Most of current machine learning-based methods for brain age gap prediction work on per-brain basis and do not consider pairwise similarities between patients, not taking into account the population as a whole.

% Graphs are static so it makes sense that most practical approaches would not use this as their model, because it prevents using applying it for new patients

% Kaufmann et al.~\cite{kaufmann2019} uses gradient boosting based techniques (XGBoost)~\cite{chen2016xgboost} for brain age gap prediction from structural magnetic resonance imaging (MRI) data. The study uses a significantly larger dataset of 45,000 people and presents separate models for female and male brain age gap prediction, without considering any pairwise similarities between individuals.

% Another framework for brain age gap prediction is based on Gaussian Processes regression, using raw T1-weighted MRI scans, segmenting them and using principal components analysis (PCA) for dimensionality reduction~\cite{cole2018brain}.
