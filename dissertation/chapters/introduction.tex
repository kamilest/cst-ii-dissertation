\chapter{Introduction}
\pagenumbering{arabic}

% The Introduction should explain the principal motivation for the project. Show how the work fits into the broad area of surrounding Computer Science and give a brief survey of previous related work. It should generally be unnecessary to quote at length from technical papers or textbooks. If a simple bibliographic reference is insufficient, consign any lengthy quotation to an appendix.

The prevalence of age-associated disease and disability is increasing with the ageing population. Increasingly common and debilitating are the neurological and neurodegenerative disorders such as Alzheimer's disease, Parkinson's disease\dots

Brain maturity and ageing depends on many complex factors depending on both genetic and postnatal environmental experience, and is associated with cellular damage, changes in brain connectivity, structural gray matter and white matter changes, morphological and functional connectivity changes, frontal lobe thinning \cite{cole2018brain, niu2019improved, franke2019ten}. This is expressed in a cognitive decline, decreased memory capacity and a variety of increasingly common neurological and neurodegenerative disorders such as Alzheimer's and dementia. With the increasing global population lifespan, those disorders are increasingly disabling and therefore there is an interest in finding the biomarkers representing the brain health status and revealing the common risk factors, giving insight into the mechanisms of ageing, potentially delaying the onset of the disease, slowing it down and reducing its prevalence. With the increased availability of brain imaging data through the population-wide studies such as the UK Biobank, and increasing computing power enabling the analysis of rich and otherwise computationally intractable datasets (including the machine learning techniques), brain ageing is becoming a popular research direction.

One biomarker for estimating the overall health of the brain is the so-called \textit{brain age}, which is a prediction of a person's \textit{chronological} (actual) age based on the neuroimaging analysis of the patient, most commonly structural magnetic resonance imaging (MRI) data. Recent studies have linked the deviations between the brain age estimate and the chronological age to the occurrence of various neurological conditions such as Alzheimer's and dementia \cite{kaufmann2019}, presumably because the overestimate of the chronological age indicates the accelerating ageing and cellular damage accumulation of the brain.


Why population graphs are a nice approach (use of multimodal data; individual patient data \textit{and} diagnoses of other patients; potential robustness to missing or noisy data of a patient as other patient can "make up" for it).

Why my work is novel (applying a new population graph framework to a different – and more challenging – problem).

\section{Related work}
\subsection{Population graphs}
Semisupervised population graphs in Parisot et al. for classifying healthy patients and patients having Mild Cognitive Impairment/Alzheimer's or autism spectrum disorder. These achieve state-of-the-art accuracies around 70\% even for a binary classification task which indicates that brain conditions are generally a complex task. Due to the different nature of the task and different evaluation metrics, this performance cannot be directly compared to the predictive power of the brain age regression task.

\subsection{Brain age prediction from neuroimaging data}
Most of current machine learning-based methods for brain age gap prediction work on per-brain basis and do not consider pairwise similarities between patients, not taking into account the population as a whole.

Graphs are static so it makes sense that most practical approaches would not use this as their model, because it prevents using applying it for new patients

Kaufmann et al. \cite{kaufmann2019} uses gradient boosting based techniques (XGBoost) \cite{chen2016xgboost} for brain age gap prediction from structural magnetic resonance imaging (MRI) data. The study uses a significantly larger dataset of 45,000 people and presents separate models for female and male brain age gap prediction, without considering any pairwise similarities between individuals.

Another framework for brain age gap prediction is based on Gaussian Processes regression, using raw T1-weighted MRI scans, segmenting them and using principal components analysis (PCA) for dimensionality reduction \cite{cole2018brain}.
