\chapter{Introduction}
\pagenumbering{arabic}

% The Introduction should explain the principal motivation for the project. Show how the work fits into the broad area of surrounding Computer Science and give a brief survey of previous related work. It should generally be unnecessary to quote at length from technical papers or textbooks. If a simple bibliographic reference is insufficient, consign any lengthy quotation to an appendix.

Brain ageing depends on many complex factors related to both genetics and postnatal environment. It is associated with cellular damage, changes in morphological and functional brain connectivity, changes in gray and white matter structure, frontal lobe thinning and other factors that result in cognitive decline, decreased memory capacity and a variety of neurological and neurodegenerative disorders such as Alzheimer's and Parkinson's diseases \cite{cole2018brain, niu2019improved, franke2019ten}. With the increasing global population lifespan, age-related brain disorders have become more common, motivating the research in biomarkers to represent an individual's brain health status, reveal common risk factors, give insight into the mechanisms of brain ageing, potentially delaying the onset of the disease, slowing it down and reducing its prevalence. With the increased availability of brain imaging data through the longitudinal population-wide studies (such as the UK Biobank), and increasing computing power (enabling the analysis of rich and otherwise computationally intractable datasets), brain ageing is becoming a popular research direction.

One biomarker for estimating the overall brain health is the so-called \textit{brain age}, defined as the \textit{apparent} age of an individual's brain compared to the typical population.

and derived from the neuroimaging analysis of the patient, most commonly magnetic resonance imaging (MRI) data. which is a \textit{prediction} of a person's chronological (actual) age .

Recent studies have linked the deviations between the brain age estimate and the true chronological age (the \textit{brain age gap}) to the occurrence of various neurological conditions such as multiple sclerosis and dementia \cite{kaufmann2019}, presumably because the overestimate of the chronological age indicates the accelerating ageing and higher cellular damage accumulation of the brain.

Population graphs, where nodes represent the neuroimaging data and edges represent some similarity metric between patients is a promising approach to analysing those datasets, as it allows to leverage both the individual subject data as well as the outcomes of similar patients, just like in diagnosing patients clinicians would look at the examples of other patients who had those particular symptoms and which disease they were diagnosed with. 

This dissertation proposes applying the (novel?) population graph paradigm to the brain age estimation task.

\section{Related work}
\subsection{Population graphs}
Semisupervised population graphs in Parisot et al. for classifying healthy patients and patients having Mild Cognitive Impairment/Alzheimer's or autism spectrum disorder. These achieve state-of-the-art accuracies around 70\% even for a binary classification task which indicates that brain conditions are generally a complex task. Due to the different nature of the task and different evaluation metrics, this performance cannot be directly compared to the predictive power of the brain age regression task.

\subsection{Brain age prediction from neuroimaging data}
Most of current machine learning-based methods for brain age gap prediction work on per-brain basis and do not consider pairwise similarities between patients, not taking into account the population as a whole.

Graphs are static so it makes sense that most practical approaches would not use this as their model, because it prevents using applying it for new patients

Kaufmann et al. \cite{kaufmann2019} uses gradient boosting based techniques (XGBoost) \cite{chen2016xgboost} for brain age gap prediction from structural magnetic resonance imaging (MRI) data. The study uses a significantly larger dataset of 45,000 people and presents separate models for female and male brain age gap prediction, without considering any pairwise similarities between individuals.

Another framework for brain age gap prediction is based on Gaussian Processes regression, using raw T1-weighted MRI scans, segmenting them and using principal components analysis (PCA) for dimensionality reduction \cite{cole2018brain}.
