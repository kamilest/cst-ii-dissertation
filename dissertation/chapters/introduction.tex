\chapter{Introduction}
\pagenumbering{arabic}

% The Introduction should explain the principal motivation for the project. Show how the work fits into the broad area of surrounding Computer Science and give a brief survey of previous related work. It should generally be unnecessary to quote at length from technical papers or textbooks. If a simple bibliographic reference is insufficient, consign any lengthy quotation to an appendix.

\section{Aims and motivation}
Why measuring brain age gaps is interesting and important.

Define the \textit{brain age gap} and its popularity in current research as a good predictor of various neurological and neurodegenerative diseases. Brain age gaps are used as biomarkers\dots

Accelerated brain ageing



Why population graphs are a nice approach (use of multimodal data; individual patient data \textit{and} diagnoses of other patients; potential robustness to missing or noisy data of a patient as other patient can "make up" for it).

Why my work is novel (applying a new population graph framework to a different – and more challenging – problem).

\section{Related work}
\subsection{Population graphs}
Semisupervised population graphs in Parisot et al. for classifying healthy patients and patients having Mild Cognitive Impairment/Alzheimer's or autism spectrum disorder. These achieve state-of-the-art accuracies around 70\% even for a binary classification task which indicates that brain conditions are generally a complex task. Due to the different nature of the task and different evaluation metrics, this performance cannot be directly compared to the predictive power of the brain age regression task.

\subsection{Brain age prediction from neuroimaging data}
To my knowledge, most of the studies on brain age gaps are per-brain basis and do not consider pairwise similarities between patients, not taking into account the population as a whole. The frameworks are also do not involve graph approaches or deep learning.

Graphs are static so it makes sense that most practical approaches would not use this as their model, because it prevents using applying it for new patients

Kaufmann et al. \cite{kaufmann2019} uses gradient boosting based techniques (XGBoost) \cite{chen2016xgboost} for brain age gap prediction from structural magnetic resonance imaging (MRI) data. The study uses a significantly larger dataset of 45,000 people and presents separate models for female and male brain age gap prediction, without considering any pairwise similarities between individuals.

Another framework for brain age gap prediction is based on Gaussian Processes regression, using raw T1-weighted MRI scans, segmenting them and using principal components analysis (PCA) for dimensionality reduction \cite{cole2018brain}.
